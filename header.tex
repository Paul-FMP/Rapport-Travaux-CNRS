% !TEX TS-program = pdflatex
% !TEX encoding = UTF-8 Unicode

% This is a simple template for a LaTeX document using the "article" class.
% See "book", "report", "letter" for other types of document.

\usepackage[T1]{fontenc}
\usepackage[french]{babel}

\usepackage[utf8]{inputenc} % set input encoding (not needed with XeLaTeX)

%%% Examples of Article customizations
% These packages are optional, depending whether you want the features they provide.
% See the LaTeX Companion or other references for full information.

%%% PAGE DIMENSIONS
\usepackage{geometry} % to change the page dimensions
\geometry{a4paper} % or letterpaper (US) or a5paper or....
\geometry{margin=1.7 cm} % for example, change the margins to 2 inches all round
% \geometry{landscape} % set up the page for landscape
%   read geometry.pdf for detailed page layout information

\usepackage{graphicx} % support the \includegraphics command and options

% \usepackage[parfill]{parskip} % Activate to begin paragraphs with an empty line rather than an indent

%%% PACKAGES
\usepackage{booktabs} % for much better looking tables
\usepackage{array} % for better arrays (eg matrices) in maths
\usepackage{paralist} % very flexible & customisable lists (eg. enumerate/itemize, etc.)
\usepackage{verbatim} % adds environment for commenting out blocks of text & for better verbatim
\usepackage{subfig} % make it possible to include more than one captioned figure/table in a single float

\usepackage{csquotes}
\usepackage{graphicx} % support the \includegraphics command and options
\usepackage{amssymb}
\usepackage{amsmath, amsfonts, bbm, dsfont}
\usepackage[all]{xy}
\usepackage{MnSymbol}
\usepackage{soul}

\usepackage{tikz-cd}
\usepackage{tikzit}
\input{tikz_style.tikzstyles}

% These packages are all incorporated in the memoir class to one degree or another...

%%% HEADERS & FOOTERS
\usepackage{fancyhdr} % This should be set AFTER setting up the page geometry
\pagestyle{fancy} % options: empty , plain , fancy
\renewcommand{\headrulewidth}{0pt} % customise the layout...
\lhead{}\chead{}\rhead{}
\lfoot{}\cfoot{\thepage}\rfoot{}

%%% SECTION TITLE APPEARANCE
\usepackage{sectsty}
\allsectionsfont{\sffamily\mdseries\upshape} % (See the fntguide.pdf for font help)
% (This matches ConTeXt defaults)

%%% ToC (table of contents) APPEARANCE
\usepackage[nottoc,notlof,notlot]{tocbibind} % Put the bibliography in the ToC
\usepackage[titles,subfigure]{tocloft} % Alter the style of the Table of Contents
\renewcommand{\cftsecfont}{\rmfamily\mdseries\upshape}
\renewcommand{\cftsecpagefont}{\rmfamily\mdseries\upshape} % No mathbf!

\usepackage[
backend=biber,
style=alphabetic,
giveninits=true,
doi=false,
url=true,
isbn=false]{biblatex}
\AtEveryBibitem{\clearfield{month}\clearlist{language}}

\renewbibmacro{in:}{}

\addbibresource{bibliography.bib}

\usepackage{amsthm}
\usepackage{tikz-cd}
\usepackage{appendix}
\usepackage{framed, color}
\usepackage[tikz]{bclogo}
\usepackage{mdframed}

\tikzcdset{row sep/normal= 0.8 cm,
column sep/normal= 0.8 cm}


\DeclareMathOperator{\flex}{\textbf{\textasciicircum}}

\newtheorem{theo}{Theorem}[section]
\newtheorem{theorem}[theo]{Théorème}
\newtheorem{prop}[theo]{Proposition}
\newtheorem{lemma}[theo]{Lemma}
\newtheorem{coro}[theo]{Corollary}
\newtheorem{hyp}[theo]{Hypothesis}
\newtheorem{claim}[theo]{Claim}

\newtheorem{recall}[theo]{Recall}

\theoremstyle{definition}
\newtheorem{fact}[theo]{Fact}
\newtheorem{defi}[theo]{Definition}
\newtheorem{definition}[theo]{Definition}
\newtheorem{notation}[theo]{Notation}

\newtheorem{rem}[theo]{Remark}
\newtheorem{remark}[theo]{Remark}

\newtheorem{ex}[theo]{Exemple}
\newtheorem{exemple}[theo]{Exemple}

\newtheorem{contre-ex}[theo]{Counter-example}






\newtheorem{motiv}[theo]{Motivation}
\newtheorem{conj}[theo]{Conjecture}
\newtheorem{conv}[theo]{Convention}
\newtheorem{constr}[theo]{Construction}
\newtheorem{convention}[theo]{Convention}
\newtheorem{construction}[theo]{Construction}


	

\newtheorem{exo}[theo]{Exercise}
\newtheorem{qst}[theo]{Question}
\newtheorem{checklist}[theo]{Checklist}


\usepackage{hyperref}
\hypersetup{
    colorlinks=true,
    linkcolor=black,
    filecolor=black,      
    urlcolor=black,
    citecolor=black,
}

\usepackage{tikz}
\usetikzlibrary{intersections}
\usepackage{quiver}

\newcommand{\brightarrow}{\mathrel{\vcenter{\hbox{\ooalign{$\rightarrow$\cr\hidewidth$|$\hidewidth}}}}}
 \newcommand{\bmrightarrow}{\mathrel{\vcenter{\hbox{\ooalign{$\rightarrow$\cr\hidewidth$\mid$\hidewidth}}}}}
  \newcommand{\bvrightarrow}{\mathrel{\vcenter{\hbox{\ooalign{$\rightarrow$\cr\hidewidth$\vert$\hidewidth}}}}}
    \newcommand{\bsmrightarrow}{\mathrel{\vcenter{\hbox{\ooalign{$\rightarrow$\cr\hidewidth$\shortmid$\hidewidth}}}}}









\newcommand{\vin}{\,\varepsilon\,}

\newcommand{\ON}{\mathbf{On}}
\newcommand{\NO}{\mathbf{No}}
\newcommand{\bP}{\mathbb{P}}
\renewcommand{\AA}{\mathcal{A}}
\newcommand{\BB}{\mathcal{B}}
\newcommand{\CC}{\mathcal{C}}
\newcommand{\DD}{\mathcal{D}}
\newcommand{\FF}{\mathcal{F}}
\newcommand{\II}{\mathcal{I}}

\DeclareMathOperator{\GG}{\mathcal{G}}
\DeclareMathOperator{\HH}{\mathcal{H}}

\newcommand{\cS}{\mathcal{S}}
\newcommand{\cT}{\mathcal{T}}

\newcommand{\M}{\mathrm{M}}
\newcommand{\N}{\mathrm{N}}
\newcommand{\T}{\mathrm{T}}

\DeclareMathOperator{\Aa}{\mathbb{A}}
\DeclareMathOperator{\Bb}{\mathbb{B}}
\DeclareMathOperator{\Dd}{\mathbb{D}}
\DeclareMathOperator{\Ee}{\mathbb{E}}
\DeclareMathOperator{\Gg}{\mathbb{G}}
\DeclareMathOperator{\Nn}{\mathbb{N}}
\DeclareMathOperator{\Qq}{\mathbb{Q}}
\DeclareMathOperator{\Cc}{\mathbb{C}}
\DeclareMathOperator{\Ff}{\mathbb{F}}
\DeclareMathOperator{\Rr}{\mathbb{R}}
\DeclareMathOperator{\Zz}{\mathbb{Z}}
\DeclareMathOperator{\Vv}{\mathbb{V}}
\newcommand{\Part}[1]{\mathcal{P}(#1)}
\newcommand{\lens}[4]{\left(\begin{array}{c}#1 \\#2  \\\end{array}\right) \leftrightarrows \left(\begin{array}{c}#3 \\#4  \\\end{array}\right)}
\newcommand{\chart}[4]{\left(\begin{array}{c}#1 \\#2  \\\end{array}\right) \rightrightarrows \left(\begin{array}{c}#3 \\#4  \\\end{array}\right)}

\DeclareMathOperator{\isom}{\simeq}
\DeclareMathOperator{\imp}{\Rightarrow}
\DeclareMathOperator{\sminus}{\backslash}
\renewcommand{\phi}{\varphi}
\DeclareMathOperator{\Aut}{Aut}


