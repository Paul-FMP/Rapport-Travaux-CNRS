\documentclass[11pt]{article} % use larger type; default would be 10pt

% !TEX TS-program = pdflatex
% !TEX encoding = UTF-8 Unicode

% This is a simple template for a LaTeX document using the "article" class.
% See "book", "report", "letter" for other types of document.

\usepackage[T1]{fontenc}
\usepackage[french]{babel}

\usepackage[utf8]{inputenc} % set input encoding (not needed with XeLaTeX)

%%% Examples of Article customizations
% These packages are optional, depending whether you want the features they provide.
% See the LaTeX Companion or other references for full information.

%%% PAGE DIMENSIONS
\usepackage{geometry} % to change the page dimensions
\geometry{a4paper} % or letterpaper (US) or a5paper or....
\geometry{margin=1.7 cm} % for example, change the margins to 2 inches all round
% \geometry{landscape} % set up the page for landscape
%   read geometry.pdf for detailed page layout information

\usepackage{graphicx} % support the \includegraphics command and options

% \usepackage[parfill]{parskip} % Activate to begin paragraphs with an empty line rather than an indent

%%% PACKAGES
\usepackage{booktabs} % for much better looking tables
\usepackage{array} % for better arrays (eg matrices) in maths
\usepackage{paralist} % very flexible & customisable lists (eg. enumerate/itemize, etc.)
\usepackage{verbatim} % adds environment for commenting out blocks of text & for better verbatim
\usepackage{subfig} % make it possible to include more than one captioned figure/table in a single float

\usepackage{csquotes}
\usepackage{graphicx} % support the \includegraphics command and options
\usepackage{amssymb}
\usepackage{amsmath, amsfonts, bbm, dsfont}
\usepackage[all]{xy}
\usepackage{MnSymbol}
\usepackage{soul}

\usepackage{tikz-cd}
\usepackage{tikzit}
\input{tikz_style.tikzstyles}

% These packages are all incorporated in the memoir class to one degree or another...

%%% HEADERS & FOOTERS
\usepackage{fancyhdr} % This should be set AFTER setting up the page geometry
\pagestyle{fancy} % options: empty , plain , fancy
\renewcommand{\headrulewidth}{0pt} % customise the layout...
\lhead{}\chead{}\rhead{}
\lfoot{}\cfoot{\thepage}\rfoot{}

%%% SECTION TITLE APPEARANCE
\usepackage{sectsty}
\allsectionsfont{\sffamily\mdseries\upshape} % (See the fntguide.pdf for font help)
% (This matches ConTeXt defaults)

%%% ToC (table of contents) APPEARANCE
\usepackage[nottoc,notlof,notlot]{tocbibind} % Put the bibliography in the ToC
\usepackage[titles,subfigure]{tocloft} % Alter the style of the Table of Contents
\renewcommand{\cftsecfont}{\rmfamily\mdseries\upshape}
\renewcommand{\cftsecpagefont}{\rmfamily\mdseries\upshape} % No mathbf!

\usepackage[
backend=biber,
style=alphabetic,
giveninits=true,
doi=false,
url=true,
isbn=false]{biblatex}
\AtEveryBibitem{\clearfield{month}\clearlist{language}}

\renewbibmacro{in:}{}

\addbibresource{bibliography.bib}

\usepackage{amsthm}
\usepackage{tikz-cd}
\usepackage{appendix}
\usepackage{framed, color}
\usepackage[tikz]{bclogo}
\usepackage{mdframed}

\tikzcdset{row sep/normal= 0.8 cm,
column sep/normal= 0.8 cm}


\DeclareMathOperator{\flex}{\textbf{\textasciicircum}}

\newtheorem{theo}{Theorem}[section]
\newtheorem{theorem}[theo]{Théorème}
\newtheorem{prop}[theo]{Proposition}
\newtheorem{lemma}[theo]{Lemma}
\newtheorem{coro}[theo]{Corollary}
\newtheorem{hyp}[theo]{Hypothesis}
\newtheorem{claim}[theo]{Claim}

\newtheorem{recall}[theo]{Recall}

\theoremstyle{definition}
\newtheorem{fact}[theo]{Fact}
\newtheorem{defi}[theo]{Definition}
\newtheorem{definition}[theo]{Definition}
\newtheorem{notation}[theo]{Notation}

\newtheorem{rem}[theo]{Remark}
\newtheorem{remark}[theo]{Remark}

\newtheorem{ex}[theo]{Exemple}
\newtheorem{exemple}[theo]{Exemple}

\newtheorem{contre-ex}[theo]{Counter-example}






\newtheorem{motiv}[theo]{Motivation}
\newtheorem{conj}[theo]{Conjecture}
\newtheorem{conv}[theo]{Convention}
\newtheorem{constr}[theo]{Construction}
\newtheorem{convention}[theo]{Convention}
\newtheorem{construction}[theo]{Construction}


	

\newtheorem{exo}[theo]{Exercise}
\newtheorem{qst}[theo]{Question}
\newtheorem{checklist}[theo]{Checklist}


\usepackage{hyperref}
\hypersetup{
    colorlinks=true,
    linkcolor=black,
    filecolor=black,      
    urlcolor=black,
    citecolor=black,
}

\usepackage{tikz}
\usetikzlibrary{intersections}
\usepackage{quiver}

\newcommand{\brightarrow}{\mathrel{\vcenter{\hbox{\ooalign{$\rightarrow$\cr\hidewidth$|$\hidewidth}}}}}
 \newcommand{\bmrightarrow}{\mathrel{\vcenter{\hbox{\ooalign{$\rightarrow$\cr\hidewidth$\mid$\hidewidth}}}}}
  \newcommand{\bvrightarrow}{\mathrel{\vcenter{\hbox{\ooalign{$\rightarrow$\cr\hidewidth$\vert$\hidewidth}}}}}
    \newcommand{\bsmrightarrow}{\mathrel{\vcenter{\hbox{\ooalign{$\rightarrow$\cr\hidewidth$\shortmid$\hidewidth}}}}}









\newcommand{\vin}{\,\varepsilon\,}

\newcommand{\ON}{\mathbf{On}}
\newcommand{\NO}{\mathbf{No}}
\newcommand{\bP}{\mathbb{P}}
\renewcommand{\AA}{\mathcal{A}}
\newcommand{\BB}{\mathcal{B}}
\newcommand{\CC}{\mathcal{C}}
\newcommand{\DD}{\mathcal{D}}
\newcommand{\FF}{\mathcal{F}}
\newcommand{\II}{\mathcal{I}}

\DeclareMathOperator{\GG}{\mathcal{G}}
\DeclareMathOperator{\HH}{\mathcal{H}}

\newcommand{\cS}{\mathcal{S}}
\newcommand{\cT}{\mathcal{T}}

\newcommand{\M}{\mathrm{M}}
\newcommand{\N}{\mathrm{N}}
\newcommand{\T}{\mathrm{T}}

\DeclareMathOperator{\Aa}{\mathbb{A}}
\DeclareMathOperator{\Bb}{\mathbb{B}}
\DeclareMathOperator{\Dd}{\mathbb{D}}
\DeclareMathOperator{\Ee}{\mathbb{E}}
\DeclareMathOperator{\Gg}{\mathbb{G}}
\DeclareMathOperator{\Nn}{\mathbb{N}}
\DeclareMathOperator{\Qq}{\mathbb{Q}}
\DeclareMathOperator{\Cc}{\mathbb{C}}
\DeclareMathOperator{\Ff}{\mathbb{F}}
\DeclareMathOperator{\Rr}{\mathbb{R}}
\DeclareMathOperator{\Zz}{\mathbb{Z}}
\DeclareMathOperator{\Vv}{\mathbb{V}}
\newcommand{\Part}[1]{\mathcal{P}(#1)}
\newcommand{\lens}[4]{\left(\begin{array}{c}#1 \\#2  \\\end{array}\right) \leftrightarrows \left(\begin{array}{c}#3 \\#4  \\\end{array}\right)}
\newcommand{\chart}[4]{\left(\begin{array}{c}#1 \\#2  \\\end{array}\right) \rightrightarrows \left(\begin{array}{c}#3 \\#4  \\\end{array}\right)}

\DeclareMathOperator{\isom}{\simeq}
\DeclareMathOperator{\imp}{\Rightarrow}
\DeclareMathOperator{\sminus}{\backslash}
\renewcommand{\phi}{\varphi}
\DeclareMathOperator{\Aut}{Aut}






\title{Rapport sur les travaux effectués}



\author{Paul $\mathrm{Wang}$}

\date{}












\begin{document}
	

	
	\setcounter{tocdepth}{2}
	\maketitle
	

	% \footnotetext[2]{\copyright \,  2023. This manuscript version is made available under the CC-BY-NC-ND 4.0 license http://creativecommons.org/licenses/by-nc-nd/4.0/}
	
	  \tableofcontents


%%%%%%%%%%%%%%%%%%



Ce rapport est divisé en deux. Dans une première partie, je présente mes travaux en théorie géométrique des modèles, en commençant par exposer les principes généraux. 


Dans la section \ref{sect_groupoids}, je donne du contexte sur la notion de revêtement en théorie des modèles, puis présente mes résultats sur la correspondance entre revêtements analysables et groupoïdes "simpliciaux" définissables, suivant une suggestion de Hrushovski, dans l'article publié suivant: 


\begin{mdframed}[backgroundcolor=yellow]
	\begin{itemize}[$\bullet$]
		\item 	\emph{Extending Hrushovski's groupoid-cover correspondence using simplicial groupoids} \cite{Wang_2021}.
	\end{itemize}
\end{mdframed}


Ensuite, dans la section \ref{sect_config_group}, j'explique la notion de configuration de groupe, rappelle l'énoncé du théorème de configuration de groupe classique (attribué à Hrushovski), et présente ma généralisation au cadre génériquement stable, de la publication

\begin{mdframed}[backgroundcolor=yellow]
	\begin{itemize}[$\bullet$]
		\item 	\emph{The group configuration theorem for generically stable types} \cite{Wang_group_config}.
	\end{itemize}
\end{mdframed}


Le théorème de configuration de groupe peut, avec des ajustements et en le combinant avec d'autres outils, être utilisé pour classifier les groupes définissables dans diverses structures. Dans la section \ref{sect_interp_groups}, je donne du contexte par rapport à cette question des groupes et corps définissables, et présente mes résultats généraux, inclus dans ma thèse \cite[Chapter 3]{Wang_thesis_2025}, ainsi que dans la prépublication

\begin{mdframed}[backgroundcolor=yellow]
	\begin{itemize}[$\bullet$]
		\item 	\emph{On groups and fields interpretable in $\mathrm{NTP}_2$ fields} \cite{Wang_definable_fields}.
	\end{itemize}
\end{mdframed}


Mes résultats généraux de classification ne s'appliquent pas à tous les exemples : certaines structures intéressantes, comme les corps valués D-henseliens, ne vérifient pas les hypothèses requises. Dans la section \ref{sect_VDF}, après avoir énoncé les résultats connus les plus importants sur ces corps, je décris mes contributions, reposant sur la stabilité générique, à l'étude des groupes et corps définissables dans ces structures, présentes dans ma thèse (pas encore soumises pour publication)

\begin{mdframed}[backgroundcolor=yellow]
	\begin{itemize}[$\bullet$]
		\item 	\emph{On groups and fields definable in D-henselian fields} \cite[Chapter 4]{Wang_thesis_2025}.
	\end{itemize}
\end{mdframed}


Dans le contexte des corps valués, la stabilité générique est en lien avec une autre notion, celle de domination stable, capturant l'idée de contrôle via le corps résiduel, surtout étudiée dans le cas des corps valués \emph{algébriquement clos}. Des généralisations à d'autres corps valués, à travers des notions de domination résiduelle, ont été développées récemment. Dans la section \ref{sect_resdom}, je présente ce contexte, ainsi que mon travail en collaboration avec Dicle Mutlu, sur l'une de ces notions, et le lien avec les groupes définissables 


\begin{mdframed}[backgroundcolor=yellow]
	\begin{itemize}[$\bullet$]
		\item 	\emph{Residually Dominated Groups in Henselian Valued Fields of Equicharacteristic Zero} \cite{Mutlu_Wang_arxiv_2025}.
	\end{itemize}
\end{mdframed}

\bigskip

Dans la seconde partie, je présente mes travaux les plus récents, en théorie catégorique des systèmes. Avant d'introduire le point de vue doublement catégorique le plus récent, je donne du contexte sur deux approches classiques, à savoir l'utilisation de catégories monoidales symétriques
et les coalgèbres d'endofoncteurs. Je donne également un aperçu des enjeux en théorie synthétique des probabilités, avant de rentrer dans le coeur du sujet, et d'expliquer mes contributions à la théorie doublement catégorique des systèmes, de la prépublication  
\begin{mdframed}[backgroundcolor=yellow]
	\begin{itemize}[$\bullet$]
		\item 	\emph{Nondeterministic Behaviours in Double Categorical Systems Theory} \cite{Wang_2025}.
	\end{itemize}
\end{mdframed}


%%%%%%%%%%%%%%%%%

\part{Théorie géométrique des modèles}

La théorie des modèles du premier ordre vise à étudier des structures et théories (algébriques, combinatoires, etc.) en s'appuyant sur des outils de logique du premier ordre -- rappelons que cela signifie que les seuls quantificateurs autorisés sont ceux portant sur des éléments; il n'y a pas de quantification sur les parties. Une théorie du premier ordre est une classe de structures définie par une collection d'axiomes du premier ordre, dans un langage donné. Dans sa version moderne, les objets centraux sont les \emph{ensembles définissables}, c'est-à-dire les collections d'éléments définies par des formules du langage. En logique classique, les ensembles définissables forment des algèbres de Boole (intersection, complémentaire, union); par dualité de Stone, l'on peut donc s'intéresser aux espaces topologiques profinis, c'est-à-dire compacts et dont les ouverts sont unions d'ouverts-fermés, associés. Ces espaces sont appelés \emph{espaces de types}, et donnent un point de vue plus "géométrique" pour l'étude des ensembles définissables -- de même que l'étude des schémas affines peut être considérée comme plus "géométrique" que l'étude des anneaux commutatifs; en fait, \emph{les espaces profinis sont des schémas affines} où le faisceau d'anneaux peut être reconstruit à partir de la topologie, via les ouverts-fermés.


Si la complexité combinatoire des ensembles définissables est faible, la théorie est considérée comme modérée. Par exemple, les \emph{théories stables}, définies dans \cite{Shelah-Thesis}, et étudiées plus en détail dans \cite{She-NIP}, sont celles où l'on ne peut définir d'ordre total infini.

\begin{exemple}
	\begin{itemize}
		\item Dans les corps algébriquement clos, munis uniquement de la structure de corps, les seuls ensembles définissables sont les \emph{constructibles}, c'est-à-dire les combinaisons booléennes de lieux d'annulation de polynômes. En particulier, on dit que la théorie \emph{élimine les quantificateurs}: même en s'autorisant à utiliser des quantificateurs, les seuls ensembles que l'on peut définir par des formules du premier ordre sont déjà définissables sans quantificateurs.
		\item A l'inverse, dans l'anneau des entiers relatifs, la complexité de quantification n'a pas de borne, et est en lien avec des questions fines de calculabilité: les ensembles définissables par une formule existentielle sont exactement ceux qui sont récursivement énumérables, c'est-à-dire pour lesquels il existe une fonction calculable qui les énumère, etc.
		\item Entre les deux, dans le corps des nombres réels, l'on peut utiliser des quantificateurs existentiels pour définir des objets tels que le disque unité fermé: $D = \lbrace (x, y) \, | \, \exists z \,\, x^2+y^2 = 1 + z^2 \rbrace$.
	\end{itemize}
\end{exemple}


L'on peut autoriser des éléments des structures considérées à être utilisés comme symboles dans les formules, ce qui permet de définir davantage d'ensembles -- par exemple, en considérant des polynômes à coefficients dans l'anneau ambiant, et pas seulement dans $\mathbb{Z}$. Des questions fines de canonicité, et de contrôle, des paramètres de définition peuvent alors se poser, analogues aux questions de corps de définition en géométrie algébrique.



Un principe général, la \emph{trichotomie de Zilber}, affirme que le comportement d'une théorie modérée dépend du type de structures algébriques définissables, c'est-à-dire, dont l'ensemble sous-jacent et les opérations sont définissables, qui y apparaissent: ou bien aucun groupe infini définissable, ou bien des groupes abéliens infinis définissables mais pas de corps infini définissable, ou bien des corps infinis définissables. Ces idées ont été formalisées dans divers contextes, à l'instar des géométries de Zariski -- Cadre visant à capturer de manière abstraite le comportement géométrique des courbes algébriques \cite{hru-zilber-zariski} -- ou des théories o-minimales \cite{omin-trichotomy} (point de vue, modèle-théorique à l'origine, sur la géométrie et l'analyse réelles, consistant à restreindre les classes de fonctions et d'ensembles afin d'exclure les comportements pathologiques). Une question connexe, au coeur de mes travaux, est celle de la \emph{classification des groupes et corps définissables}, ou interprétables dans une structure donnée, un ensemble interprétable étant un quotient d'ensemble définissable par une relation d'équivalence définissable.


\section{Revêtements et groupoïdes définissables}\label{sect_groupoids}

Lorsque l'on étudie la structure fine des ensembles définissables dans une théorie ou structure donnée, la notion de \emph{revêtement} apparaît assez vite. Etant donnée une structure $\mathbb{U}$, un \emph{revêtement de} $\mathbb{U}$ \emph{via une nouvelle sorte} $S$ est une structure $\mathbb{U}'$, dont l'ensemble sous-jacent est l'union disjointe de celui de $\mathbb{U}$ et de $S$, telle que tout sous-ensemble de $\mathbb{U}$ définissable dans $\mathbb{U}'$ avec paramètres possiblement dans $S$ est déjà définissable avec paramètres dans $\mathbb{U}$. On dit aussi que $\mathbb{U}$ est \emph{stablement plongée} dans $\mathbb{U}'$. Un fait important, qui explique la terminologie, est que, dans une structure stable, tout ensemble définissable est stablement plongé.


\subsection{Revêtements internes et groupoïdes de liaison}

Une classe de revêtements intéressante est celle des \emph{revêtements internes}. Dans une structure ambiante donnée, un ensemble définissable $X$ est \emph{interne} à un ensemble définissable $Y$ s'il existe une surjection définissable $Y^k \twoheadrightarrow X$, pour un entier $k$.


Un point crucial est qu'une telle surjection définissable peut demander des paramètres (pour être définie) supplémentaires, par rapport à ceux utilisés pour définir $X$ et $Y$.

\begin{exemple}
	Considérons la théorie des corps différentiellement clos de caractéristique nulle: il s'agit de la théorie des corps algébriquement clos de caractéristique nulle munis d'une dérivation, i.e. d'un endomorphisme additif vérifiant la formule de Leibniz, telle que toute équation différentielle polynomiale en une variable, non dégénérée, a "beaucoup" de solutions. Notons-la $\mathrm{DCF}_0$. Cette théorie est stable, et élimine les quantificateurs.



	Le corps des constantes, défini par l'équation différentielle $x' = 0$, est algébriquement clos, et sa structure induite est réduite aux opérations de corps. Considérons également l'ensemble $X$ défini par l'équation $x'=1$. Alors, par linéarité, tout élément de $X$ fournit une bijection définissable entre $X$ et le corps des constantes. Cependant, sans paramètres, il n'y a pas de fonctions définissables intéressante entre les deux.


	Un objet mesurant cette dépendance en le choix d'une solution particulière est le \emph{groupe de liaison}, ici isomorphe au groupe additif des constantes, qui agit sur $X$.

\end{exemple}


Cette dépendance en les paramètres est contrôlée par un \emph{groupoïde} de liaison, introduit par Hrushovski dans \cite{HRUSHOVSKI_2012}, qui est une version intrinsèque des \emph{groupes de liaison}, étudiés en théorie de la stabilité, notamment par Zilber \cite{Zilber_1980} et Poizat \cite{Poizat_1983}, ce dernier développant une généralisation modèle-théorique de la théorie de Galois des corps, incluant les quotients définissables. Rappelons qu'un groupoïde est donné par une collection d'objets et de morphismes inversibles, avec une loi de composition associative et unitaire; la principale différence avec la notion de groupe est que la composée de deux morphismes n'est définie que si le domaine de l'un est égale à l'image de l'autre. L'intérêt, ici comme dans d'autres domaines\footnote{L'on pensera par exemple à la différence entre groupe fondamental et groupoïde fondamental en topologie algébrique, la première notion supposant de choisir un point base.}, est \emph{l'absence de choix} à faire.



\subsection{Questions}

Etant donnée une structure $\mathbb{U}$ et un ensemble $A$ définissable sans paramètres, un revêtement $(\mathbb{U}, S)$ de $\mathbb{U}$ est dit \emph{1-analysable au-dessus de A} s'il existe
une fonction surjective définissable sans paramètres $f: S \rightarrow A$, dont les fibres $S_a$ sont internes à $\mathbb{U}$.

\begin{exemple}\label{ex_DCF_1-an}
	Dans la théorie des corps différentiellement clos, l'ensemble $X$ défini par l'équation différentielle ${(\frac{x'}{x})' = 0}$ et la condition $x \neq 0$, donne un revêtement $1$-analysable sur le corps des constantes $C$, via la fonction de dérivation logarithmique $x \in X \mapsto \frac{x'}{x} \in C$. En effet, pour tout $\lambda \in C$, l'ensemble $X_{\lambda} = \lbrace x\neq 0 \, | \, x' = \lambda \cdot x \rbrace$ est interne à $C$: toute solution définit, par multiplication, une bijection entre $X_{\lambda}$ et $C\setminus \lbrace 0 \rbrace$.

	Notons que $X$ est un sous-groupe multiplicatif du corps différentiel ambiant, et s'inscrit dans la suite exacte courte de groupes abéliens suivante: $1 \rightarrow C^{\times} \rightarrow X \rightarrow C \rightarrow 0$.
\end{exemple}




\begin{itemize}
	\item Les constructions de Hrushovski peuvent-elles s'étendre au cas des revêtements $1$-analysables ?
	\item Si oui, peut-on donner un critère pour qu'un revêtement $1$-analysable soit en fait interne ?
	\item Y a-t-il des versions uniformes en familles ?

\end{itemize}




\subsection{Travaux existants}

\begin{itemize}
	\item Dans \cite{HRUSHOVSKI_2012}, Hrushovski démontre, entre autres choses, une correspondance, pour toute structure $\mathbb{U}$, entre groupoïdes définissables \emph{sans paramètres} dans $\mathbb{U}$ et revêtements internes de $\mathbb{U}$ à une sorte supplémentaire -- modulo une petite hypothèse technique, signalée par Haykazyan et Moosa. L'idée cruciale est que les morphismes du groupoïde définissable sont des bijections définissables témoignant de l'internalité de la sorte supplémentaire.
	\item Dans \cite{Haykazyan_Moosa_2018}, Haykazyan et Moosa démontrent d'une part que la correspondance de Hrushovski provient d'une équivalence de catégories, pour des bonnes notions de morphismes de revêtements internes -- à savoir, des fonctions entre les sortes supplémentaires, définissables dans un revêtement interne commun -- et de morphismes de groupoïdes définissables. Fidèles au principe de ne pas faire de choix arbitraire, ils définissent un morphisme de groupoïdes comme, essentiellement, un \emph{profoncteur} définissable, c'est-à-dire une catégorie étendant l'union disjointe en ajoutant des morphismes entre les objets des deux groupoïdes; ils demandent par ailleurs des propriétés importantes d'homogénéité.

	Enfin, ils étendent la correspondance au cas de revêtements $1$-analysables à fibres indépendantes -- condition forte, exprimant l'absence d'interaction, au niveau des ensembles définissables, entre les fibres, c'est-à-dire qu'ils donnent une version "uniforme le long de familles définissables" de la construction. 

	\item Dans sa thèse, Jimenez démontre un résultat uniforme \cite[Theorem 3.1.3]{Jimenez_thesis_2020} le long de familles de revêtements internes qu'il appelle \emph{paires relativement internes}. Il construit également des \emph{Delta} groupoïdes -- notion similaire, mais combinatoirement plus simple, à celle de groupoïdes \emph{simpliciaux}, qu'il utilise pour caractériser l'internalité pour ses paires relativement internes, via une condition de dégénérescence. Cependant, comme expliqué dans la discussion \cite[Fin de la Section 3.2]{Jimenez_thesis_2020}, ses constructions ne répondent pas à la question de non-canonicité des paramètres.
		%En collaboration avec Pillay et Jaoui, il développe la notion d'\emph{internalité relative uniforme}, qui est équivalente à la dégénérescence des Delta groupoïdes

\end{itemize}




\subsection{Contributions}
Dans l'article \cite{Wang_2021}, je définis une notion de \emph{groupoïde simplicial} définissable -- qui devrait plutôt s'appeler Delta groupoïde, comme dans \cite{Jimenez_thesis_2020} -- au-dessus d'un ensemble définissable $A$. Un tel objet est donné par:

\begin{enumerate}
	\item Une famille $\mathcal{G}_n$, $n \geq 1$, de groupoïdes définissables sans paramètres, où, pour tout $n$, le groupoïde $\mathcal{G}_n$ admet une partition définissable indexée par l'ensemble des parties de $A$ de cardinal $n$.
	\item Des morphismes $\iota_{n, m}: \mathcal{G}_n \rightarrow \mathcal{G}_m$, pour $n < m$, entre groupoïdes définissables, au sens de Haykazyan et Moosa \cite{Haykazyan_Moosa_2018}, c'est-à-dire donnés par une extension définissable de la catégorie définissable $\mathcal{G}_n \sqcup \mathcal{G}_m$, i.e. des collections de morphismes des objets de $\mathcal{G}_n$ vers ceux de $\mathcal{G}_m$.


	Je demande que ces morphismes de groupoïdes $\iota_{n, m}$ soient compatibles avec les partitions de $\mathcal{G}_n$ et $\mathcal{G}_m$, que les fonctions à l'intérieur soient \emph{injectives}, et que l'égalité $\iota_{n,m} \circ \iota_{k, n} = \iota_{k, m}$ soit vraie pour $k < n <m$, au sens de la composition de morphismes de groupoïdes (voir \cite{Haykazyan_Moosa_2018}).
\end{enumerate}


Je définis ensuite une notion de \emph{morphisme entre groupoïdes simpliciaux sur A}.



Je démontre alors les résultats suivants, pour toute $\mathbb{U}$ et tout $A$:

\begin{itemize}
	\item Il existe une correspondance entre revêtements $1$-analysables sur $A$, vérifiant des conditions de finitude et de plongement stable, et groupoïdes simpliciaux sur $A$ vérifiant la propriété de l'union disjointe, i.e. tels que les objets en haut degré sont unions disjointes d'objets en bas degré.
	\item Cette correspondance provient d'une équivalence de catégories, pour des catégories de revêtements $1$-analysables et de groupoïdes simpliciaux \emph{constituées d'isomorphismes}.
	\item Les constructions se généralisent en supprimant la condition de finitude du côté des revêtements; les groupoïdes simpliciaux obtenus ne sont alors que pro-définissables, i.e. obtenus par produits possiblement infinis d'ensembles définissables.
	\item Je calcule ensuite les objets correspondants pour l'exemple \ref{ex_DCF_1-an}, ainsi que pour la suite exacte courte de groupe abéliens $0 \rightarrow {(\mathbb{Z}/ 2 \mathbb{Z})}^{\mathbb{N}} \rightarrow {(\mathbb{Z}/ 4 \mathbb{Z})}^{\mathbb{N}} \rightarrow {(\mathbb{Z}/ 2 \mathbb{Z})}^{\mathbb{N}} \rightarrow 0$.

	Je montre, pour ces deux exemples, que les groupoïdes simpliciaux restreints aux degrés $\leq 3$ suffisent à capturer la structure totale des revêtements: seule la loi de groupe compte, et elle est déjà encodée dans les degrés $\leq 3$.
\end{itemize}

\begin{mdframed}{\textbf{Idées pour construire des groupoïdes simpliciaux:}}
	\begin{itemize}
		\item Suivant la suggestion de Hrushovski dans \cite{HRUSHOVSKI_2012}, étant donné un revêtement $1$-analysable $(\mathbb{U}, S)$ sur un ensemble définissable $A$, je considère, pour toute partie finie $\overline{c} \subset A$, le revêtement \emph{interne} défini par l'union des fibres $S_{\overline{c}} = \bigcup\limits_{a \in \overline{c}} S_a$. La construction de Hrushovski s'applique alors, j'obtiens un groupoïde définissable avec les paramètres $\overline{c}$, que je note $\mathcal{G}_{\overline{c}}$.
		\item Je montre que, pour tout $n$, la construction ci-dessus peut se faire uniformément en $\overline{c} \subset A$ de cardinal $n$; j'obtiens donc des groupoïdes définissables $\mathcal{G}_n$.
		\item Les morphismes $\iota_{n, m}: \mathcal{G}_n \rightarrow \mathcal{G}_m$ sont construits à partir des inclusions $S_{\overline{c}} \hookrightarrow S_{\overline{c}'}$, pour $\overline{c} \subset \overline{c}' \subset A$ finies.
	\end{itemize}

Cette construction est proche de celle de Jimenez; une différence est que mes morphismes de groupoïdes sont des profoncteurs, et non pas simplement des foncteurs. C'était déjà un point crucial dans les travaux de Haykazyan et Moosa, qui faisait fonctionner leurs constructions.
Une autre différence est que Jimenez n'autorise, dans ses groupoïdes, qu'un objet par fibre, restriction absente de mes constructions; c'est probablement la raison pour laquelle mes groupoïdes "simpliciaux" sont intrinsèques, et pas les siens. Cependant, Jimenez s'intéresse à des automorphismes entre fibres distinctes, ce qui n'est pas mon cas: chaque (ensemble fini de) fibre correspond à une composante connexe distincte.
\end{mdframed}


\begin{mdframed}{\textbf{Idées pour construire des revêtements 1-analysables:}}
	Etant donné un groupoïde simplicial définissable $\mathcal{G}$ au-dessus d'un $A$, je construis un revêtement $1$-analysable sur $A$ comme suit:

	\begin{itemize}
		\item Je construis par induction transfinie un \emph{système cohérent d'inclusions}, ce qui correspond à choisir un objet par composante connexe en chaque degré, ainsi qu'une famille cohérente d'injections parmi les fonctions des $\iota_{n, m}$, entre ces objets. 
		\item Etant donné un système cohérent d'inclusions, je construit une sorte supplémentaire $S$, consistant à recoller les copies des objets choisis le long des injections de la famille. Par construction, la sorte $S$ vient avec une surjection définissable $S \rightarrow A$, dont les fibres sont internes à $\mathbb{U}$.
		\item Pour montrer que cela ne dépend pas des choix faits, je construis par induction, étant donnés deux systèmes cohérents d'inclusions, une "famille cohérente d'isomorphismes partiels" entre eux, constituée de morphismes de $\mathcal{G}$. Une telle famille s'assemble alors en un isomorphisme de structures entre les extensions construites.
		\item Pour montrer que les structures que je construis sont bien des revêtements, j'applique un critère d'extension d'automorphismes \cite[Appendix, Lemma 1]{ChaHru-ACFA}, et j'utilise de nouveau une construction par induction transfinie.
	\end{itemize}


	Cette construction généralise celle de Haykazyan et Moosa \cite{Haykazyan_Moosa_2018}, avec des conditions de compatibilité supplémentaires, automatiques pour eux vu leur hypothèse d'indépendance des fibres.
\end{mdframed}



\section{Configurations de groupe génériquement stables}\label{sect_config_group}

Le principe du théorème de configuration de groupe est de \emph{détecter la présence d'un groupe définissable} \emph{à partir de données combinatoires}. Ce résultat s'inscrit dans le domaine de la \emph{théorie géométrique des modèles}, dont on rappelle qu'elle consiste à étudier les théories du premier ordre à travers les structures algébriques définissables qui y apparaissent. 



\subsection{Contexte}

Une configuration de groupe est la donnée de $6$ éléments vérifiant des propriétés d'indépendance et de dépendance bien spécifiques; l'exemple essentiel est le suivant: soit $G$ un groupe définissable dans une théorie stable -- i.e. où aucune partie définissable n'induit une relation d'ordre total infini; on rappelle qu'il s'agit d'une hypothèse forte de modération combinatoire -- par exemple la théorie des corps algébriquement clos. Soient $g_1$, $g_2$, $g_3$ des éléments \emph{génériques} de $G$, formant une \emph{famille indépendante}: par exemple, si $G = \mathrm{GL}_n(\mathbb{C})$, la conjonction de ces deux conditions correspond à demander que les $g_i$ soient en position générale, l'ensemble des triplets de $\mathrm{GL}_n(\mathbb{C})$ ne vérifiant pas cette propriété est négligeable pour la mesure de Lebesgue restreinte à ${\mathrm{GL}_n(\mathbb{C})} \times {\mathrm{GL}_n(\mathbb{C})} \times {\mathrm{GL}_n(\mathbb{C})} $. Alors, le diagramme ci-dessous est une configuration de groupe: 



\begin{center}
\begin{tikzpicture}
\coordinate (b3) at (0,0) ;
\coordinate (b2) at (0,-1) ;
\coordinate (b1) at (0,-2) ;
\coordinate (a2) at (1,-0.5) ;
\coordinate (a1) at (2,-1) ;
\coordinate (a3) at (1,-1) ;


\draw (b3) -- (a1);
\draw (b3) -- (b1);
\draw (b2) -- (a1);
\draw (b1) -- (a2);

%\path [name intersections={of=CD1 and BE1, name=(f1)}];

\draw (b3) node [above] {\(g_2 \cdot g_1\)};
\draw (b2) node [left] {\(g_{2}\)};
\draw (b1) node [left] {\(g_{1}\)};
\draw (a2) node [right] {\(g_3\)};
\draw (a1) node [right] {\(g_1 \cdot g_3\)};
\draw (a3) node [below] {\(g_2 \cdot g_1 \cdot g_3\)};

\end{tikzpicture}
\end{center}

En effet, les deux propriétés définissant la notion de configuration de groupe \emph{régulière}\footnote{Il existe en effet une définition, et un théorème, pour des \emph{actions de groupe} plus générales qu'une action par translation d'un groupe sur lui-même; nous l'ignorerons ici, malgré son importance.} sont les suivantes:

\begin{itemize}
	\item Sur le diagramme, tout triplet de points non alignés forme une \emph{famille indépendante}, en un sens général fourni par la théorie de la stabilité.
	\item Pour toute ligne du diagramme, chaque point est \emph{algébrique} sur les deux autres, i.e. est \emph{d'orbite finie} sous l'action des automorphismes de la structure ambiante fixant les deux autres éléments; de manière équivalente, il existe une formule ayant un nombre fini de solutions, parmi lesquelles le point en question, utilisant comme paramètres les deux autres éléments.
\end{itemize}



L'énoncé du théorème de configuration de groupe pour les théories stables, attribué à Hrushovski, est le suivant: 

\begin{theorem}
	Pour toute configuration de groupe (régulière) dans une théorie stable, il existe un groupe type-définissable $\Gamma$, variante légèrement plus générale, où l'on autorise des intersections infinies d'ensembles définissables, et trois éléments $g_1$, $g_2$, $g_3$ du groupe $\Gamma$, génériques et indépendants, tels que la configuration de groupe initiale et celle de $\Gamma$ construite à partir des $g_i$ sont équivalentes, c'est-à-dire interalgébriques point à point.


	De plus, \emph{un tel groupe est essentiellement unique}: étant données deux configurations de groupe équivalentes construites à partir d'éléments génériques indépendants pour deux groupes définissables $\Gamma$ et $\Gamma'$, il existe une isogénie virtuelle\footnote{Autrement dit, un sous-groupe normal de $\Gamma \times \Gamma'$ pour lequel les projections vers $\Gamma$ et $\Gamma'$ sont à fibres finies, et d'images d'indice fini.} définissable entre $\Gamma$ et $\Gamma'$.

\end{theorem}


Autrement dit, les exemples présentés plus haut sont essentiellement les seuls: dans une théorie stable, tout sextuplet d'éléments vérifiant les propriétés d'indépendance et d'algébricité ci-dessus provient de génériques indépendants d'un groupe définissable. Ce résultat a des conséquences remarquables: par exemple, l'existence de types réguliers localement modulaires implique l'existence de groupes abéliens infinis type-définissables avec des génériques réguliers. En d'autres termes, Hrushovski a utilisé un théorème de configuration de groupe pour \emph{prouver l'existence de groupes avec certaines propriétés}.


\subsection{Travaux existants}
La seule variante du théorème de configuration de groupe démontrée, jusque-là, hors du cadre des théories stables, de Ben Yaacov-Tomasic-Wagner \cite{group-config-simple}, repose sur une hypothèse moins forte de \emph{simplicité} de la théorie ambiante, au prix d'une conclusion moins forte également: le groupe obtenu n'est que "presque hyper-définissable".

\subsection{Contributions}
Dans l'article publié \cite{Wang_group_config}, je démontre une généralisation du théorème de configuration de groupe originel, en supposant uniquement que le sextuplet considéré définit un \emph{type génériquement stable}, c'est-à-dire que les suites de copies indépendantes de ce sextuplet se comportent comme si elles étaient dans une théorie stable. L'énoncé est essentiellement le même, à ceci près que je donne un contrôle plus précis sur les paramètres/éléments utilisés dans la construction, qui peut être utile pour des applications du théorème. La structure de la preuve est similaire.

Au niveau technique, l'essentiel du travail a consisté à utiliser de manière précise les propriétés des types génériquement stables, en particulier la notion d'indépendance fournie par la théorie des modèles abstraite, qui se comporte \emph{presque} aussi bien que dans le cas des théories stables. La subtilité technique qui a demandé le plus d'efforts est que l'on ne sait pas en général (voir \cite[Question 4.1]{Conant_Gannon_Hanson_2025}, qui donne une réponse positive pour les théories $\mathrm{NTP}_2$) si la concaténation de deux copies indépendantes d'un type génériquement stable, aussi appelée \emph{produit tensoriel} ou \emph{produit de Morley}, est elle-même génériquement stable !



Pour ce qui est des applications, au vu des résultats obtenus dans des contextes stables en utilisant le théorème de configuration de groupe originel, cette généralisation est un pas vers le développement de résultats analogues reposant sur les types génériquement stables.







\section{Groupes et corps interprétables dans des théories de corps}\label{sect_interp_groups}

Dans \cite{Wang_interp_groups_2025}, j'étudie la structure des groupes et des corps définissables/interprétables, à la fois dans des cadres abstraits et pour des exemples explicites de théories de corps valués.



\subsection{Questions}
Etant donnée une théorie de corps (en logique du premier ordre) $T$:
\begin{itemize}
	\item Que peut-on dire des groupes définissables dans $T$ ? Peut-on les comparer définissablement à des "groupes algébriques en coordonnées" ? Rappelons que les "groupes algébriques en coordonnées" sont toujours définissables, car les "variétés en coordonnées" et fonctions rationnelles le sont, puisque les opérations de corps font partie de la structure. 
	\item Quid des corps définissables dans $T$ ? Sont-ils réduits aux extensions finies du corps ambiant ? En effet, celles-ci sont toujours définissables, en choisissant des bases au sens linéaire et en calculant en coordonnées.
	\item Que dire si l'on généralise ces questions au cas des groupes et corps \emph{interprétables}, i.e. si l'on autorise les quotients d'ensembles définissables par relations d'équivalence définissables ?
\end{itemize}

\subsection{Théories de corps (valués)}

Pour la suite, un corps valué est donné par un corps $K$, un groupe abélien ordonné\footnote{L'on demande que l'ordre soit total, et compatible avec la loi de groupe.} $\Gamma$, que l'on notera multiplicativement ici, et un morphisme de groupes surjectif $|\cdot |: K^{\times} \rightarrow \Gamma$, appelé norme, vérifiant l'inégalité ultramétrique suivante: $|x+y| \leq \max (|x|, |y|)$. Usuellement, la norme est étendue en ajoutant un élément absorbant à $\Gamma$, égal à la norme de l'élément $0$. On appelle $\Gamma$ le groupe de valeurs.

\begin{exemple} Les structures suivantes sont des corps valués:
	\begin{itemize}
		\item 	Pour tout nombre premier $p$, et tout réel $r$ tel que $0 < r < 1$, le corps des nombres rationnels $\mathbb{Q}$, muni de la norme $p$-adique $x \mapsto r^{v_p(x)}$, où $v_p(x)$ désigne la valuation $p$-adique de $x$. Le groupe de valeurs est $r^{\mathbb{Z}}$, isomorphe à $\mathbb{Z}$. A isomorphisme près, cette structure ne dépend pas du choix de $r$.
		\item Pour tout nombre premier $p$, et tout réel $r$ avec $0 < r <1$, le complété de $\mathbb{Q}$ selon la norme $p$-adique, noté $\mathbb{Q}_p$. On l'appelle le corps des nombres $p$-adiques, et, à ismorphisme près, il ne dépend pas du choix de $r$. 
		\item Pour tout corps $K$, le corps des fractions rationnelles $K(t)$, avec une norme associée à la valuation $t$-adique.
		\item Pour tout corps $K$, le corps des séries de Laurent $K((t))$, avec une norme $t$-adique, qui est le complété de $K(t)$.
		\item Pour tout corps $K$ et tout groupe abélien ordonné $\Gamma$, le corps de séries de Hahn $K((t^{\Gamma}))$, dont les éléments sont les fonctions de $\Gamma$ dans $K$ à support bien ordonné, pour l'ordre induit par $\Gamma$; la somme est calculée terme à terme, et le produit est défini par convolution.

	\end{itemize}

\end{exemple}



Dans un corps valué, l'inégalité ultramétrique entraîne que la boule unité fermée $\mathcal{O}= \lbrace x \, : \, |x| \leq 1 \rbrace$ est un anneau local, dont l'idéal maximal $\mathfrak{m}$ est la boule ouverte, $\mathfrak{m}= \lbrace x \, : \, |x| < 1 \rbrace$. Le quotient $\mathcal{O} / \mathfrak{m}$ est appelé \emph{corps résiduel}.


\begin{mdframed}{\textbf{Définition:}}
	Une théorie de corps (valué) est une théorie avec un corps (valué) définissable marqué\footnote{De même qu'un espace topologique pointé est un espace topologique avec un point marqué.}.
\end{mdframed}


Un point important est que le corps (valué) définissable en question peut avoir de la structure additionnelle; l'enjeu est de comprendre comment étudier ces cas sans nécessairement passer par une analyse détaillée de ladite structure additionnelle. 

\subsection{Travaux existants}
\begin{itemize}
	\item Le premier résultat, dû à Hrushovski, inspiré par les idées de \cite{Wei-GpCh}, affirme que tout groupe définissable dans un corps algébriquement clos pur est définissablement isomorphe à un groupe algébrique en coordonnées. Cela utilise un théorème de "\emph{group chunk}", qui est une manière de reconstruire un groupe définissable à partir de données génériques, similaire au théorème de configuration de groupe.
	
	Notons que "définissable" et "interprétable" sont équivalents dans les corps algébriquement clos: on dit que ceux-ci \emph{éliminent les imaginaires}.
	\item Le résultat de Hrushovski a été utilisé par Poizat \cite{Poi-GenNS}, pour montrer que tout corps définissable dans un corps algébriquement clos pur est définissablement isomorphe à celui-ci. S'appuyant également sur ces idées, il a été prouvé dans \cite[Proposition 2.5]{Pil-OMinGp}, que tout groupe définissable dans une structure o-minimale est naturellement un groupe topologique, que tout corps définissable dans ce contexte est soit réel clos, soit algébriquement clos, et que, dans le cas du corps pur des nombres réels, les groupes définissables sont des groupes de Lie. Ces résultats ont ensuite été adaptés, dans \cite[Theorems A and C]{HruPil-GpPFF}, au cas des corps réel et p-adiques, où la topologie est prise en compte, et au cas des corps pseudo-finis, qui sont les corps infinis vérifiant toutes les propriétés du premier ordre communes aux corps finis. Pillay \cite[Corollary 4.2]{Pillay1997SomeFQ} et Suer \cite[Theorem 3.36]{Suer2007ModelTO} ont, quant à eux, traité le cas des corps différentiellement clos de caractéristique nulle, i.e. munis d'une dérivation pour laquelle les équations différentielles algébriques ont "suffisamment de solutions".
	
	%pour montrer que tout groupe de Nash sur un corps réel ou p-adique est localement isomorphe à l'ensemble des points d'un groupe algébrique, et que tout groupe définissable dans un corps pseudo-fini -- c'est-à-dire un corps infini vérifiant toutes les propriétés du premier ordre communes aux corps finis -- est virtuellement isogène à l'ensemble des points rationnels d'un groupe algébrique. En ce qui concerne les corps différentiellement clos de caractéristique nulle, i.e. munis d'une dérivation, et pour lesquels les équations différentielles algébriques ont "suffisamment de solutions", Pillay \cite[Corollary 4.2]{Pillay1997SomeFQ} a prouvé que tout groupe définissable se plonge dans un groupe algébrique, et Suer \cite[Theorem 3.36]{Suer2007ModelTO} a montré que les seuls corps définissables infinis, à isomorphisme définissable près, sont le corps différentiel ambiant et le corps des constantes.
	
	\item Le théorème le plus général jusqu'ici est \cite[Theorem 2.19]{Stabilizers-NTP2}, de Montenegro, Onshuus et Simon, qui traite des groupes définissablement moyennables -- c'est-à-dire admettant une mesure de probabilité finiment additive, définie sur l'algèbre des ensembles définissables, invariante à gauche -- dans toutes les théories de corps parfaits ayant des propriétés modèles-théoriques raisonnables (qui sont $\mathrm{NTP_2}$ et où les clôtures algébriques au sens de la théorie des modèles et au sens des corps coïncident). 
	
	\item Dans le contexte plus spécifique des corps valués, sous certaines hypothèses techniques, la classification des corps interprétables a été réalisée dans \cite[Theorem 7.1]{fields_interp_in_various_val_fields} : les seuls corps interprétables sont les extensions finies du corps ambiant ou du corps résiduel. D'autres résultats incluent les travaux réalisés dans \cite{ppp_def_groups_codf}, qui prouvent que les groupes définissables dans divers corps différentiels peuvent être définissablement plongés dans des groupes algébriques, et \cite[Theorem 1]{semisimple_groups_interp_in_various_val_fields}, montrant que les groupes définissablement semi-simples interprétables dans des enrichissements de corps valués algébriquement clos, réels clos ou  p-adiquement clos, sont virtuellement isogènes à des produits de groupes linéaires sur le corps valué et sur le corps résiduel. Dans le cas spécifique des corps valués algébriquement clos purs, la classification des corps interprétables a été initialement prouvée dans \cite{HruRK-MetaGp}, en utilisant une notion appelée métastabilité, qui se concentre sur les types génériquement stables. Enfin, dans le cadre de la 1-h-minimalité, une notion visant à capturer l'idée de géométrie modérée dans des contextes non archimédiens, il est montré dans \cite{AcoHass-1hmin} que, comme dans le cas réel, les groupes définissables admettent une structure de Lie, et que les seuls corps définissables sont les extensions finies du corps ambiant. 


\end{itemize}

\subsection{Contributions}

%Les travaux de Halevi, Hasson et Peterzil traitent des groupes interprétables sans nécessiter une élimination complète des imaginaires, une caractéristique remarquable. Ils sont également assez précis, puisque les homomorphismes de groupes qu'ils définissent sont des isogénies virtuelles, c'est-à-dire proches d'être des isomorphismes. La principale limitation, pour ainsi dire, est que leurs outils nécessitent des hypothèses assez fortes, et ne couvrent donc que certains cas. De manière similaire, l'approche géométrique d'Acosta L\'{o}pez et Hasson produit des résultats précis, par exemple en excluant les sous-corps définissables du corps ambiant. En revanche, Peterzil, Pillay et Point construisent des plongements définissables de groupes définissables dans des groupes algébriques, dont les images peuvent être petites, pour de grandes classes de corps différentiels enrichis. Cependant, ils ne traitent pas des imaginaires, tout comme Montenegro, Onshuus et Simon.

\begin{itemize}
	\item Mon travail est similaire à \cite[Theorem 2.19]{Stabilizers-NTP2} : je démontre d'abord un théorème purement abstrait de construction de morphisme de groupes définissables \cite[Théorème 3.1.47]{Wang_thesis_2025}, faisant intervenir deux théories $T_0$ et $T_1$, où $T_0$ est superstable (condition légèrement plus forte que la stabilité), $T_1$ est NIP (condition plus faible que la stabilité, couvrant les théories o-minimales, notamment le corps des réels, bon nombre de théories de corps valués, dont les corps p-adiques, etc.) et $T_0$ est une "restriction" de $T_1$. 
	
	
	Une notion clé est celle d'un ensemble définissable \(X\) qui \emph{ne voit pas} un autre ensemble définissable \(Y\) : cela signifie essentiellement que toutes les fonctions définissables \(X \rightarrow Y\) ont une image finie. Le résultat est alors le suivant: \emph{tout groupe définissablement moyennable dans la théorie NIP admet un morphisme de groupes définissable vers un groupe de la théorie superstable, dont le noyau ne voit pas les ensembles définissables de la théorie superstable.}
	
	\item Ensuite, en appliquant ce théorème abstrait pour $T_0$ la théorie des corps algébriquement clos, j'obtiens un résultat très général \cite[Théorème 3.2.2]{Wang_thesis_2025}  pour des théories de corps \emph{algébriquement bornés} -- c'est-à-dire où les clôtures algébriques au sens de la théorie des modèles et de la théorie des corps coïncident -- et je peux couvrir plus de cas que \cite[Theorem 2.19]{Stabilizers-NTP2} : sous des hypothèses modérées, \emph{tous les groupes \emph{interprétables}\footnote{Le theorème \cite[Theorem 2.19]{Stabilizers-NTP2} ne traite que les cas des groupes \emph{définissables} et noyaux finis.} définissablement moyennables admettent un morphisme définissable vers un groupe algébrique sur le corps ambiant, dont le noyau ne voit pas le corps.} Autrement dit, ces groupes interprétables définissablement moyennables admettent des suites exactes courtes, où le terme de droite est un sous-groupe d'un groupe algébrique, et le terme de gauche ne voit pas le corps.
	\item De même, en prenant pour $T_0$ la théorie des corps différentiellement clos, je démontre des résultats analogues \cite[Théorème 3.3.2]{Wang_thesis_2025} pour les groupes interprétables, pour des classes de corps différentiels assez larges.
	\item Pour ce qui est des corps interprétables, je démontre \cite[Corollaire 3.2.19]{Wang_thesis_2025}, sous hypothèses modérées, la dichotomie suivante: \emph{un corps interprétable infini admet un plongement définissable dans une extension finie du corps ambiant, ou bien ne voit pas ce dernier, auquel cas on le qualifie de \emph{purement imaginaire\footnote{Ce qui est en général le cas pour le corps résiduel d'un corps valué henselien, par exemple.}}}. 
	
	Cela repose sur le résultat intermédiaire \cite[Proposition 3.2.7]{Wang_thesis_2025} suivant: étant donnés deux corps infinis définissables $F$ et $K$, \emph{dans une théorie arbitraire}, si le groupe de transformations affines de $F$ se plonge définissablement dans un groupe algébrique sur $K$, alors le corps $F$ se plonge définissablement dans une extension finie de $K$. 
	\item Je traite ensuite des exemples spécifiques: classes générales de corps valués henseliens \cite[Théorèmes 3.3.32 et 3.3.33]{Wang_thesis_2025} recouvrant les exemples déjà connus dans la littérature, et corps valués différentiellement clos, qui sont les corps de caractéristique nulle munis d'une valuation et d'une dérivation qui "n'interagissent pas": toutes les configurations permises par l'algèbre sont réalisées, en particulier les équations différentielles algébriques non triviales ont des ensembles de solutions denses pour la topologie de la valuation. Pour ces derniers, je démontre \cite[Théorème 3.3.20]{Wang_thesis_2025} que les seuls corps infinis interprétables sont le corps valué, son corps des constantes, et le corps résiduel.

\end{itemize}



Ces résultats englobent dans un cadre général la grande majorité des exemples étudiés précédemment dans la littérature, et des preuves également; le fait qu'ils portent sur les groupes \emph{interprétables} permet des applications à l'étude des imaginaires, i.e. des quotients définissables. Par ailleurs, les théorèmes les plus abstraits \cite[Théorème 3.1.47]{Wang_thesis_2025} \cite[Théorème 3.1.30]{Wang_thesis_2025} peuvent s'appliquer à des cadres autres que des théories de corps.

%\section{Contribution incluse dans la thèse: Classification des anneaux définissables dans les corps algébriquement clos}





\section{Corps valués D-henseliens et types génériquement stables}\label{sect_VDF}

Dans les corps valués différentiellement clos, la dérivation a un comportement générique par rapport à la valuation : toutes les équations différentielles non triviales possèdent des ensembles denses de solutions. Bien que cette hypothèse facilite l'étude par des méthodes de théorie des modèles, on peut vouloir considérer des corps valués différentiels où la dérivation est continue pour la topologie de la valuation, comportement plus satisfaisant d'un point de vue géométrique. Comme montré dans \cite{Sca-DValF}, il existe toute une famille de théories complètes du premier ordre intéressantes pour les corps valués différentiels d'équicaractéristique $0$, avec des dérivations $1$-Lipschitz ; ces corps sont parfois appelés $D$-Henséliens. Je me suis concentré sur le cas le plus générique, à savoir celui où le corps résiduel, avec dérivation induite, est un corps différentiellement clos; je note cette théorie $VDF$. Des exemples de structures vérifiant ces axiomes sont donnés par les corps de séries de Hahn $K((t^{\Gamma}))$, où $\Gamma$ est un groupe abélien ordonné divisible, le corps résiduel $K$ est un corps différentiellement clos, et la dérivation est construite à partir de celle de $K$ en dérivant terme à terme les coefficients. 






\subsection{Travaux existants}
\begin{itemize}
	\item Dans \cite{Sca-DValF}, Scanlon donne une axiomatisation de $VDF$, montre qu'elle est complète, et prouve un résultat d'élimination des quantificateurs, i.e. donne une description de la structure des ensembles définissables; en fait, il démontre ces résultats pour une classe générale de théories de corps valués avec dérivation $1$-Lipschitz.
	\item Dans \cite{Rid-VDF}, Rideau-Kikuchi démontre des résultats plus fins, dont une description des imaginaires, i.e. des quotients d'ensembles définissables par relations d'équivalence définissables, et la métastabilité, qui correspond grosso modo à la présence de nombreux types génériquement stables -- dont on rappelle qu'ils ont de très bonnes propriétés de régularité, d'un point de vue modèle-théorique.
\end{itemize}

Concernant les types génériquement stables dans les corps valués algébriquement clos, et la métastabilité, les travaux de Hrushovski et Loeser \cite{HruLoe} montrent la richesse de la notion: ils construisent des analogues modèle-théoriques d'espaces de Berkovitch, dont les points sont des types génériquement stables -- en fait stablement dominés, ce qui est équivalent ici -- et construisent ensuite des comparaisons avec les espaces de Berkovitch usuels, ce qui leur permet de démontrer des résultats sur ces derniers, notamment l'existence de rétractions par déformation sur des complexes simpliciaux, à un niveau de généralité plus grand que ce qui était connu jusque-là.



\subsection{Questions}

Les questions que je considère sont les mêmes que pour \cite{Wang_interp_groups_2025}, à savoir: 
\begin{itemize}
	\item Que peut-on dire des groupes définissables dans la théorie $VDF$ ? Ceux qui n'utilisent pas d'imaginaires se plongent-ils dans des groups algébriques ?
	\item Que dire des corps définissables ?
\end{itemize}



\subsection{Contributions}
Dans le dernier chapitre de ma thèse \cite[Chapitre 4]{Wang_thesis_2025}, j'étudie les questions ci-dessus. Un point crucial est que les résultats de la section précédente ne s'appliquent pas pour la théorie $VDF$ : un obstacle majeur est que la clôture algébrique dans $VDF$ est possiblement plus grande que celle induite par la théorie des corps différentiellement clos : en raison de la condition $1$-Lipschitz, certaines équations différentielles ont une solution unique, pour des raisons similaires au théorème de point fixe de Picard, et aux résultats d'unicité des solutions d'équations différentielles ordinaires (un exemple est donné dans \cite[Proposition 3.1]{Rid-VDF}). 

Cependant, une propriété plus faible est vérifiée : tout élément du corps valué qui est algébrique -- au sens de la théorie des modèles, i.e. qui vérifie une formule ayant un nombre fini de solutions -- sur un ensemble de paramètres est \emph{différentiellement algébrique} sur celui-ci, c'est-à-dire qu'il est solution d'une équation différentielle algébrique non triviale. Ainsi, en travaillant « à données différentiellement algébriques près », j'effectue des constructions similaires à celles qui précèdent. L'idée est de suivre une variante du théorème de configuration de groupe de Hrushovski\footnote{voir par exemple \cite{Bouscaren1989TheGC}}. Cependant, en raison de difficultés techniques liées à cette variante, j'utilise une hypothèse supplémentaire sur le générique du groupe : je demande qu'il soit génériquement stable et \emph{orthogonal à tous les types différentiellement algébriques}, ce qui correspond à la \emph{stabilité par changement de base le long d'extensions} \emph{différentiellement algébriques}. L'exemple typique est le point/type générique de l'anneau de valuation, dont l'image dans le corps résiduel est le point/type différentiellement transcendant. Ainsi, je démontre les résultats suivants :

\begin{itemize}
	\item \cite[Theorem 4.1.32]{Wang_thesis_2025} Si $G$ est un groupe définissable, vivant dans le corps valué, qui admet un point/type générique génériquement stable orthogonal à tous les types différentiellement algébriques, alors il admet un morphisme définissable vers un groupe algébrique, dont le noyau est différentiellement algébrique. 
	\item \cite[Theorem 4.1.37]{Wang_thesis_2025} Si $F$ est un corps définissable, vivant dans le corps valué, qui possède un sous-anneau définissable $R$ de même degré de transcendance différentiel non nul, où $R$ admet un générique, additif et multiplicatif, génériquement stable et orthogonal aux types différentiellement algébriques, alors $F$ est définissablement isomorphe au corps valué différentiel ambiant.
\end{itemize}



\emph{Ces résultats sont, à ma connaissance, les premiers obtenus dans un cadre où les hypothèses usuelles sur la clôture algébrique ne sont pas vérifiées}; il s'agit donc d'un premier pas vers une étude plus fine des structures de ce genre. Par exemple, bien que les preuves n'utilisent pas à proprement parler la métastabilité de la théorie $VDF$, les bonnes propriétés des types génériquement stables sont cruciales, ce qui confirme, si besoin était, l'importance de ces derniers. 


Une autre classe de structures intéressantes où ces idées pourraient s'adapter est celle des corps valués avec automorphismes, qu'ils soient des isométries \cite{AzgvdD}, ou bien contractants \cite{Dor_Halevi_w_Kaplan_2025} pour la valuation. En effet, dans ces structures, de même que pour $VDF$, certaines équations faisant intervenir l'automorphisme ont une unique solution.


\section{Domination résiduelle dans les corps valués henseliens}\label{sect_resdom}

La notion de \emph{domination stable}, brièvement mentionnée plus haut, a été introduite par Haskell, Hrushovski et Macpherson \cite{hhm}, pour étudier les corps valués algébriquement clos. 
Le point de départ est le constat suivant: les structures et théories stables ont de très bonnes propriétés de régularité modèle-théoriques; cependant, bon nombre de théories et structures intéressantes n'étant pas stables, on ne peut appliquer telle quelle la théorie de la stabilité. 
Une idée est alors de considérer la \emph{partie stable} d'une structure donnée, qui est la "plus grande sous-structure stable", et d'essayer de contrôler des comportements dans la structure totale via leur trace sur la partie stable. 



Par exemple, dans les corps valués algébriquement clos, la partie stable est, à peu de choses près, le corps résiduel, qui est un corps algébriquement clos \emph{pur stablement plongé} : la structure induite par le corps valué est exactement la structure de corps, et rien de plus. L'on comprend que seuls certains types, certaines extensions, pourront être bien contrôlé(e)s par la partie stable: dans un corps valué, les extensions \emph{ramifiées} (qui font grossir le groupe de valeurs) ou \emph{immédiates} (qui ne font grossir ni le corps résiduel, ni le groupe de valeurs) ont peu de chances d'être comprises via le corps résiduel.


On dit qu'un type est \emph{stablement dominé} si \emph{le comportement de ses extensions génériques est contrôlé par ce qu'elles induisent sur la partie stable}. L'exemple canonique, pour les corps valués algébriquement clos, est celui du type d'une \emph{extension monogène purement résiduelle d'un corps sphériquement complet} (analogue fort de la complétude dans le cadre général des corps valués ultramétriques, où l'on demande que toute chaîne de boules a un point d'accumulation; cette condition est utile si la valuation n'est pas discrète). La notion a ensuite été généralisée à celle de \emph{domination résiduelle}, dans des contextes divers \cite{EHM19} \cite{ResDom2} \cite{Vic22} \cite{KRV24}.

\subsection{Travaux existants}


\begin{itemize}
	\item Les travaux de Haskell, Hrushovski et Macpherson \cite{hhm} contiennent des résultats fondamentaux sur la domination stable, entre autres, des propriétés de changement de base pour les types stablement dominés, et la métastabilité de la théorie des corps valués algébriquement clos. Un résultat particulièrement important pour la suite est le suivant: 
	

	Dans une théorie quelconque, si $G$ est un groupe définissable admettant un générique stablement dominé $p$, alors il existe une famille de groupes définissables dans la partie stable $\mathfrak{g}_i$, et de morphismes définissables $f_i: G \rightarrow g_i$, tels que le type $p$ est stablement dominé par la famille $(f_i)$ -- ce qui signifie que le contrôle des extensions décrit plus haut n'a besoin que de l'information contenue dans ces fonctions.


	
	\item Plus récemment, Cubides Kovacsics, Rideau-Kikuchi et Vicar\'{i}a \cite{Kovacsics_Rideau-Kikuchi_Vicaria_2025} ont défini une notion plus générale de \emph{domination résiduelle} et montré l'équivalence, sous hypothèses techniques assez faibles, entre cette notion et la domination stable au sens des corps valués algébriquement clos, y compris pour des imaginaires/quotients définissables. Il s'agit du résultat le plus complet à ce jour.




\end{itemize}




\subsection{Résultats principaux}
Dans \cite{Mutlu_Wang_arxiv_2025}, nous démontrons les résultats suivants pour les corps valués henseliens d'équicaractéristique nulle:

\begin{itemize}
	\item La domination résiduelle, pour les extensions régulières de type fini, est caractérisée par la conjonction de la propriété de la base séparée -- l'existence de bases pour lesquelles les normes de combinaisons linéaires se calculent explicitement en fonction des normes des coefficients -- et de l'absence de ramification. Voir \cite[Corollary 2.18]{Mutlu_Wang_arxiv_2025}.
	\item La domination résiduelle au sens du corps valué henselien ambiant équivaut à la domination stable au sens de sa clôture algébrique, munie de l'unique valuation étendant la sienne, par henselianité, c'est-à-dire au sens de la théorie des corps valués algébriquement clos. Voir \cite[Theorem 2.19]{Mutlu_Wang_arxiv_2025}. Ce résultat, moins général que celui de \cite{Kovacsics_Rideau-Kikuchi_Vicaria_2025}, a été obtenu indépendamment, avec des méthodes plus algébriques et plus simples. Il nous est utile pour ce qui suit.
	\item Supposons que la théorie considérée est $\mathrm{NTP}_2$ (condition de modération assez faible, couvrant un grand nombre d'exemples). Si $G$ est un groupe définissable, vivant dans le corps valué, possédant un générique résiduellement dominé, alors il existe un groupe algébrique sur le corps résiduel $\mathfrak{g}$ et un morphisme de groupes définissable surjectif $f$ de $G$ vers $\mathfrak{g}$, tel que tous les génériques de $G$ sont dominés par $f$. En particulier, tous les génériques de $G$ sont résiduellement dominés. Voir \cite[Theorem 3.6]{Mutlu_Wang_arxiv_2025}.
	\item Supposons maintenant que la théorie considérée est NIP (notion plus restrictive que $\mathrm{NTP}_2$, mais encore assez large, comprenant les corps p-adiques), et soit $G$ un groupe définissable ayant un générique résiduellement dominé, comme précédemment. Alors, il existe un groupe $G_1$, définissable et stablement dominé au sens des corps valués algébriquement clos, et un morphisme de groupes définissables $\iota$, de noyau fini, allant de $G$ dans $G_1$, tel que l'image de tout générique de $G$ est générique dans $G_1$ au sens des corps valués algébriquement clos.
	
	De plus, le groupe $G_1$ et le morphisme $\iota$ sont essentiellement universels, et essentiellement fonctoriels vis-à-vis des morphismes surjectifs de groupes définissables. Voir \cite[Theorem 3.7 et Proposition 3.8]{Mutlu_Wang_arxiv_2025}.
\end{itemize}



\subsection{Contributions personnelles}

Pour cet article, une bonne partie des idées ont été le résultat d'interactions avec ma co-auteure Dicle Mutlu; les contributions que l'on pourrait m'attribuer sont les suivantes:

\begin{itemize}
    \item En exploitant les propriétés fines de la généricité dans le cadre $\mathrm{NTP}_2$, je démontre que tout générique d'un groupe ayant \emph{un} générique résiduellement dominé est lui-même résiduellement dominé.

	\item J'utilise le théorème du stabilisateur \cite[Theorem 2.15]{Stabilizers-NTP2} pour construire des morphismes de groupes définissables à partir de données génériques, ce qui me permet de démontrer l'universalité et la fonctorialité des constructions.
	
\end{itemize}






\part{Théorie catégorique des systèmes}

La théorie catégorique des systèmes (voir par exemple l'introduction de \cite{Libkind_Myers_2025}) vise à donner un cadre théorique unifié pour l'étude de systèmes dynamiques, les aspects techniques reposant sur des notions de théorie des catégories. Plusieurs directions ont émergé de ce champ de recherches au fil du temps; les deux plus importantes, en ce qui concerne ma recherche, sont l'approche fondée sur les \emph{catégories monoidales symétriques}, possiblement avec de la structure additionnelle (voir \ref{subs_smc}), et le \emph{point de vue coalgébrique} sur les systèmes (voir \ref{subs_coalg}). En effet, mon travail s'inscrit dans la nouvelle branche appelée \emph{théorie doublement catégorique}, ou \emph{doublement opéradique}, des systèmes \cite{Myers_2021} \cite{Libkind_Myers_2025}, qui vise à combiner les deux approches susmentionnées dans un seul cadre; il repose également sur la théorie synthétique des probabilités. Il me semble utile de commencer par rappeler en quoi consistent les points de vue "monoidal symétrique" et "coalgébrique", ainsi que la théorie synthétique des probabilités, avant d'expliquer les enjeux et questions en théorie doublement catégorique des systèmes, et mes contributions.

\section{Vers un point de vue doublement catégorique}

\subsection{Approche monoidale symétrique}\label{subs_smc}
Une catégorie monoidale symétrique est une structure contenant des objets, des morphismes entre ces objets, une opération de produit sur les objets (muni de symétries canoniques), et des opérations de composition séquentielle et parallèle sur les morphismes, vérifiant des propriétés algébriques raisonnables (associativité, etc.). L'intérêt pour la représentation de systèmes dynamiques est le suivant: si l'on peut représenter les espaces d'entrées ou de sorties possibles comme des objets,  les systèmes comme des morphismes, et que l'on dispose effectivement d'opérations de compositions parallèle et séquentielle de systèmes, alors \emph{la théorie des catégories monoidales symétriques fournit automatiquement une syntaxe et des outils de raisonnement}. Les plus emblématiques sont les \emph{diagrammes de cordes}\footnote{Appelés \emph{string diagrams} en anglais; voir par exemple la page du nLab sur le sujet: \url{ncatlab.org/nlab/show/string+diagram}.}, qui permettent des calculs rigoureux fondés sur des manipulations graphiques des diagrammes. 


Une des applications les plus marquantes est le \emph{ZX-calculus} \cite{Danos_Kashefi_Panangaden_2007}, utilisé pour représenter des calculs quantiques à base de qbits, dont les vertus pédagogiques, découlant vraisemblablement de l'approche graphique et de la relative simplicité du langage, ont été testées expérimentalement \cite{Coecke_Kissinger_Gogioso_Dündar-Coecke_Puca_Yeh_Waseem_Pothos_Pfaendler_Wang-Mascianica_et_al._2025}. 


L'accent mis sur la structure des systèmes composites permet de démontrer des résultats de manière modulaire: par exemple, dans \cite{Broadbent_Karvonen_2023}, Broadbent et Karvonen formalisent le paradigme de \emph{cryptographie universellement composable}, en définissant une notion flexible et structurée de \emph{modèles d'attaques}, et de \emph{sécurité} de processus contre de tels modèles d'attaque. Dans leur cadre, fondé sur la notion de catégorie monoidale symétrique, ils démontrent des résultats de \emph{préservation des propriétés de sécurité par composition} \cite[Theorem 4.6]{Broadbent_Karvonen_2023}. Ils utilisent également le langage graphique des catégories monoidales symétriques pour redémontrer, de manière rigoureuse \emph{et} visuelle, certains résultats de base en cryptographie \cite[Sections 6, 7 et 8]{Broadbent_Karvonen_2023}.


\subsection{Point de vue coalgébrique}\label{subs_coalg}
Pour illustrer le point de vue coalgébrique sur les systèmes, utilisons l'exemple des machines de Moore, notion qui généralise celle d'automate fini déterministe. 

\begin{mdframed}{\textbf{Machines de Moore comme coalgèbres}}


	Une machine de Moore, dans les ensembles, est définie par des ensembles d'états, entrées et sorties, notés respectivement $S$, $I$, et $O$, et des fonctions de sortie $S \rightarrow O$, et de mise à jour d'état $S \times I \rightarrow S$. 
	
	
	L'observation est alors que, pour $I$ et $O$ fixés, la donnée d'une machine de Moore est équivalente à la donnée d'un ensemble $S$, et d'une fonction $S \rightarrow O \times S^I$, c'est-à-dire d'un objet $S$ de la catégorie des ensembles, muni d'un morphisme $S \rightarrow F(S)$, où $F$ désigne le foncteur $X \mapsto O \times X^I$. Autrement dit, une machine de Moore est une coalgèbre pour le foncteur $F$. Cette observation n'est pas restreinte au cas des machines de Moore, mais se décline sur un grand nombre d'exemples. Voir \cite[Section 3]{Rutten_2000}. 

\end{mdframed}



Partant de ce principe, l'on peut ensuite définir, suivant par exemple Rutten \cite{Rutten_2000}, \footnote{Il s'agit davantage d'\emph{identifier une structure commune} à des théories de systèmes, d'\emph{organiser l'information}, que de mener une étude axiomatique approfondie à partir de telles définitions.} un système comme étant une coalgèbre pour un foncteur $F: \CC \rightarrow \CC$, étant entendu que la notion dépend fortement de $F$, en particulier, le choix de $F$ détermine une "interface" commune. Une \emph{simulation} d'un système vers un autre est alors simplement un morphisme de coalgèbres; dans la plupart des exemples, cette notion traduit l'idée de "morphisme de comparaison entre espaces d'états, \emph{agissant identiquement sur les interfaces}, compatible avec les dynamiques des systèmes considérés". Des notions plus symétriques, de \emph{bisimulation} et \emph{bisimilarité}, peuvent ensuite être définies \cite{Staton_2009}. %Le point de vue coalgébrique a donné lieu à un grand nombre de développements \cite{Geuvers_Jacobs_2021} \cite{Smithe_2023}, notamment au niveau logique \cite{Kurz} \cite{Gallardo_Viglizzo_2024} \cite{Fábregas_Palomino_Frutos-Escrig_2024}.


L'intérêt de ce point de vue est de permettre un traitement uniforme: par exemple, dans \cite{Wissmann_Dorsch_Milius_Schröder_2020}, un algorithme générique de minimisation de systèmes est présenté, i.e. qui calcule un système simple ayant même comportement observable que le système de départ. Sous hypothèses techniques modérées, les auteurs développent une version  optimisée, ayant les mêmes performances asymptotiques que les meilleurs algorithmes spécifiques. Leur point de vue général leur permet de traiter des systèmes qui combinent des types de transitions différents: non-déterminisme, probabilités, etc.  


\subsection{Théorie synthétique des probabilités et du nondéterminisme, catégories de Markov}\label{subs_th_synth_proba}

Avant de présenter la théorie doublement catégorique des systèmes, je souhaite m'attarder sur un élément important au niveau "monoidal symétrique", à savoir la \emph{théorie synthétique des probabilités}, fondée sur la notion de catégorie de Markov \cite[Definition 2.1]{FRITZ-MarkovCats}. L'idée est d'axiomatiser le comportement des \emph{noyaux de Markov}, ou noyaux de probabilités, entre espaces mesurables, qui à chaque élément de la source associent une mesure de probabilité sur l'espace d'arrivée, mesure qui "varie de manière mesurable" (voir \cite[Section 1]{Perrone2022MarkovCA} et \cite{Giry_1982}). Comme les noyaux de Markov sont munis d'opération de compositions parallèle et séquentielle, \emph{le cadre monoidal symétrique est pertinent}. Cependant, de la structure additionnelle existe, en l'occurrence, pour tout espace mesurable $X$, l'unique noyau de Markov $del_X$, de $X$ vers l'espace singleton, ainsi que la fonction mesurable diagonale $copy_X: X \rightarrow X \times X$. Abstrayant à partir de cet exemple:

\begin{mdframed}{\textbf{Définition:}}
Une \emph{catégorie de Markov} est une catégorie monoidale symétrique où tout objet $X$ est muni de morphismes $del_X: X \rightarrow 1$ et $copy_X: X \rightarrow X \otimes X$, telle que ces morphismes vérifient un certain nombre de propriétés algébriques simples.
\end{mdframed}



Notons que ces axiomes admettent des modèles variés, dont certains représentent des notions de nondéterminisme \emph{possibilistes}, c'est-à-dire où l'incertitude correspond à des \emph{ensembles d'issues possibles}, sans mesure de probabilité pour la quantifier. L'opération consistant à associer à une mesure de probabilité (raisonnable) son \emph{support} induit alors un 
foncteur entre les catégories de Markov correspondantes, représentant l'oubli d'information. Ces considérations ont une importance pour les applications à l'étude des systèmes d'IA: un certain nombre de résultats expérimentaux utilisant des méthodologies moins robustes, ou avec des analyses statistiques peu fournies, peuvent être considérés comme \emph{possibilistes}, et ainsi avoir des conclusions en adéquation avec leurs méthodes.
Un champ d'applications important de ces idées est en sémantique des langages de programmation probabilistes \cite{HigherOrderQBS} \cite{LazyPPL} \cite{Lavore_Felice_Román_2025} \cite{Liell-Cock_Staton_2025}.



Un point important, comme expliqué par Tobias Fritz dans l'introduction de son article fondateur \cite{FRITZ-MarkovCats}, est que cette approche est \emph{axiomatique et synthétique}: il s'agit d'étudier le \emph{comportement} des objets à un niveau relativement abstrait, plutôt que de s'appuyer sur des descriptions (ensemblistes par exemple) très précises. Parmi les exemples d'approches synthétiques fructueuses en mathématiques, on pourra penser à la théorie des \emph{catégories abéliennes}, qui donne un cadre efficace pour traiter des questions d'algèbre homologique, à la théorie des \emph{infini-cosmos} \cite{Riehl_Verity_2022}, qui axiomatise non pas les infini-catégories, mais les 
univers dans lesquels elles interagissent en tant qu'objets, appelés infini-cosmos -- ce point de vue permet un traitement \emph{indépendant du modèle combinatoire} choisi pour définir ce qu'est une infini-catégorie, ce qui a son importance étant donné la pluralité des modèles existants, et à la théorie synthétique de la courbure de Ricci \cite{Villani_2006} \cite{Lott_Villani_2009}. 


Fritz mentionne un certain nombre d'avantages pour cette approche.
D'une part, le \emph{plus grand niveau de généralité} et la \emph{modularité}: un théorème synthétique donné, avec liste d'hypothèses bien identifiée, pourra en général s'instancier dans divers cadres, ce qui évite d'avoir à refaire des preuves similaires. D'autre part, l'approche \emph{haut niveau} permet, en faisant abstraction de détails d'implémentation, de manipuler des objets et constructions possiblement complexes plus facilement, de même que pour les catégories abéliennes, ou les infini-catégories (stables), et de faciliter la découverte de nouveaux résultats théoriques.


Ainsi, un certain nombre de résultats de théorie des probabilités classique ont été redémontrés dans ce cadre synthétique, enrichi le cas échéant: loi du 0-1 de Kolmogorov et Hewitt-Savage \cite{Fritz2020infiniteproducts}, théorème de Blackwell-Sherman-Stein \cite{Fritz_Gonda_Perrone_Fjeldgren_Rischel_2023}, loi forte des grands nombres \cite{Fritz_Gonda_Lorenzin_Perrone_Mohammed_2025}, théorème de décomposition ergodique \cite{Moss_Perrone_2023}... Avec ce point de vue, des choix judicieux pour les définitions et les axiomes permettent de simplifier les preuves abstraites, l'idée étant que retrouver les énoncés usuels demande ensuite de vérifier que les axiomes abstraits sont satisfaits; il y a donc une division plus claire entre ces vérifications, qui peuvent s'avérer techniques, et les preuves générales.



A noter que l'utilisation de méthodes synthétiques en statistique mathématique est antérieur à ces travaux récents, par exemple chez Dawid \cite{Dawid_1979} \cite{Dawid_1980} \cite{Dawid_2001}, Lauritzen et Massa \cite{Massa_Lauritzen_2010}, et McCullagh \cite{McCullagh_2002}.


\subsection{Théorie doublement catégorique des systèmes}
Comme mentionné plus haut, la théorie doublement catégorique des systèmes vise à combiner un point de vue sur la \emph{composition} de systèmes et des notions de \emph{comparaisons}, ou \emph{simulations généralisées} entre systèmes. L'un des objectifs est de donner des principes et méthodes, étayés par la théorie, pour la \emph{modélisation collaborative}; les projets les plus saillants dans ce domaine sont ModelCollab \cite{UofS-CEPHIL/modelcollab_2025}  et CatColab \cite{Carlson_2024}.

Au niveau technique, cela repose sur l'utilisation de \emph{catégories doubles}\footnote{Voir \url{ncatlab.org/nlab/show/double+category} ou \cite[Introduction]{Dawson_Pare_1993} pour plus de détails}. 

\begin{mdframed}{\textbf{Catégories doubles et simulations généralisées:}}
\begin{itemize}
	\item 	Une catégorie double (stricte) est définie par deux catégories sur la même classe d'objets, parfois appelées les catégories \emph{verticale} et \emph{horizontale}, ainsi que la donnée de $2$-cellules, pour tout quadruplet de morphismes formant les côtés parallèles d'un carré. Les $2$-cellules sont parfois appelées \emph{carrés}, et sont munies d'opération de composition verticale et horizontale, associatives unitaires, satisfaisant la loi d'échange. 

	\item 	Etant donnés des systèmes $S$ et $T$, vus comme cellules de dimension $1$ dans la catégorie double en jeu, un \emph{comportement} de forme $S$ dans $T$, ou \emph{simulation généralisée} de $S$ vers $T$, ou \emph{morphisme de systèmes} de $S$ vers $T$, est défini comme étant une $2$-cellule. Ainsi, le principe est de construire des catégories doubles où l'une des directions encode la composition de systèmes, et l'autre les simulations généralisées entre interfaces, et entre systèmes. \emph{La loi d'échange représente alors une condition de compatibilité cruciale entre composition de systèmes et composition de simulations généralisées}.

	
	\item Une idée importante est que chaque système \emph{représente} un type de comportement, et une question (cf \cite[Fin de la Section 3.5]{DCST-book}) est de trouver des systèmes (nécessairement simples) représentant les notions de \emph{trajectoires}, nondéterministes en l'occurrence.
	\item 	Remarquons que la notion de catégorie double est plus générale que celle, plus courante, de $2$-catégorie\footnote{Voir \url{ncatlab.org/nlab/show/2-category}}; la différence n'est pas anecdotique, car les catégories doubles permettent \emph{considérer des simulations généralisées entre systèmes n'ayant pas la même interface}.

\end{itemize}

	




	

\end{mdframed}








\subsection{Questions}
\begin{mdframed}{\textbf{Questions:}}
\begin{itemize}
	\item Comment définir une notion de "simulation généralisée" entre systèmes nondéterministes (stochastiques, possibilistes, etc.) pouvant inclure du nondéterminisme, dans un cadre doublement catégorique ?
	\item Comment répondre à la première question de manière paramétrique/fonctorielle en la notion de nondéterminisme ?
	\item Comment capturer, dans une définition uniforme, les systèmes à temps discret et à temps continu ?
\end{itemize}

\end{mdframed}



\subsection{Travaux existants}

Voici une sélection de travaux pertinents vis-à-vis de nos questions.

\begin{itemize}

	\item Dans \cite{Katis_Sabadini_Walters_1997}, Katis, Sabadini et Walters construisent des bicatégories (notion de "2-catégorie non nécessairement stricte") de processus, où les $2$-cellules représentent des simulations entre processus, i.e. des comparaisons entre états internes, à comportement observable fixé. 
	

	\item Dans \cite{Baez_Courser_2018}, une catégorie (monoidale symétrique) double est construite, où une direction encode des processus de Markov ouverts, donc des systèmes probabilistes, et l'autre des fonctions de \emph{coarse-graining}. La principale restriction est que ces fonctions sont déterministes. Un autre point est la focalisation sur les compositions séquentielle et parallèle, aux dépens des schémas de composition plus généraux.
	\item Dans \cite{Lavore_Felice_Román_2025}, les systèmes à temps discret sont représentés par des suites finies de fonctions nondéterministes $M_{n-1} \times X_1 \times \cdots \times X_n \rightarrow M_n \times Y_1 \times \cdots \times Y_n$, compatibles avec les projections oubliant les dernières coordonnées; ici $M_n$ représente la mémoire à l'instant $n$, les $X_i$ correspondent aux entrées, et les $Y_i$ aux sorties. Les structures obtenues sont des \emph{catégories monoidales symétriques avec feedback}. Cet article n'aborde pas la question $2$-catégorique, mais des notions similaires à \cite{Katis_Sabadini_Walters_1997} existent sans doute.

	\item Dans \cite{DCST-book} et \cite{Libkind_Myers_2025}, les constructions utilisent la notion de \emph{monade commutative} pour le nondéterminisme; elles donnent bien des catégories doubles, mais la notion de simulation généralisée est trop restrictive, comme expliqué dans \cite[Fin de la section 3.5]{DCST-book}.

\end{itemize}

\subsection{Contributions}

Dans \cite{Wang_2025}, je construis, étant donnés une catégorie de Markov avec lois conditionnelles (version synthétique des lois conditionnelles usuelles), notée $\CC$, et un graphe orienté acyclique $\GG$, une théorie de systèmes dynamiques, en un sens très proche de ce qui est appelé "module de systèmes" dans \cite{Libkind_Myers_2025}, qui répond à la question posée dans \cite[Fin de la Section 3.5]{DCST-book}, i.e. qui autorise des trajectoires, et plus généralement des comportements, "vraiment nondéterministes". 


Cette construction permet de capturer des classes d'exemples précédemment non couvertes, ou seulement restreinte aux comportements déterministes, par la théorie doublement catégorique des systèmes:

\begin{itemize}
	\item Les systèmes (ouverts) gouvernés par des Equations Différentielles Stochastiques.
	\item Les automates nondéterministes.
	\item Les processus de décision de Markov partiellement observables (i.e. processus de Markov pouvant interagir avec un environnement extérieur).
\end{itemize}



Au niveau technique, mes constructions reposent sur les idées suivantes:
\begin{mdframed}{\textbf{Idées:}}
	
\begin{enumerate}
	\item La catégorie de Markov $\CC$ avec lois conditionnelles représente la notion de nondéterminisme de la théorie de systèmes à construire.
	\item Le graphe $\GG$ représente la notion de temps considérée; pour des raisons propres au nondéterminisme -- précisément, le fait qu'une loi sur un produit contienne plus d'information que la donnée de ses marginales, et par absence d'espaces de fonctions, je traite le temps de manière externe à $\CC$. En effet, de même que \cite{Lavore_Felice_Román_2025}, je vois les systèmes, et morphismes de systèmes, comme des familles de "fonctions nondéterministes", indexées par le graphe $\GG$, compatibles avec des fonctions de restriction oubliant les derniers instants.
	\item L'existence de lois conditionnelles dans $\CC$ permet de définir la composée verticale des $2$-cellules, ce qui correspond à \emph{construire des trajectoires jointes de systèmes composés}, en faisant une \emph{hypothèse d'indépendance conditionnelle des composants relativement à l'information aux interfaces}. 
	\item A partir de l'idée du point ci-dessus, l'essentiel du travail technique consiste à vérifier les propriétés algébriques requises; associativité et échange demandent le plus d'efforts.
	\item Afin de simplifier la rédaction, ainsi que l'étude future de la fonctorialité des constructions vis-à-vis de $\CC$ et de $\GG$, je construis d'abord des \emph{catégories triples}, où la dimension supplémentaire sert à gérer le temps, avant d'en déduire les catégories doubles voulues comme catégories de foncteurs "de source $\GG$" à valeurs dans ces catégories triples.
\end{enumerate}
\end{mdframed}


Une observation intéressante est que le gros des calculs repose sur l'utilisation des propriétés formelles de l'indépendance conditionnelle, et que lesdites propriétés sont très similaires aux propriétés des notions d'indépendance utilisées dans mes travaux précédents, en théorie des modèles. Il ne s'agit pas d'une coïncidence: d'après l'article \cite{Yaacov_2013}, l'indépendance conditionnelle de variables aléatoires (à valeurs dans un espace métrique) est une instance, en théorie des modèles continue, des notions d'indépendance générales présentes dans mes travaux en théorie des modèles.


\printbibliography[
heading=bibintoc,
title={Bibliographie}
]

    
\end{document}