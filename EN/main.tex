\documentclass[11pt]{article} % use larger type; default would be 10pt

% !TEX TS-program = pdflatex
% !TEX encoding = UTF-8 Unicode

% This is a simple template for a LaTeX document using the "article" class.
% See "book", "report", "letter" for other types of document.

\usepackage[T1]{fontenc}
\usepackage[french]{babel}

\usepackage[utf8]{inputenc} % set input encoding (not needed with XeLaTeX)

%%% Examples of Article customizations
% These packages are optional, depending whether you want the features they provide.
% See the LaTeX Companion or other references for full information.

%%% PAGE DIMENSIONS
\usepackage{geometry} % to change the page dimensions
\geometry{a4paper} % or letterpaper (US) or a5paper or....
\geometry{margin=1.7 cm} % for example, change the margins to 2 inches all round
% \geometry{landscape} % set up the page for landscape
%   read geometry.pdf for detailed page layout information

\usepackage{graphicx} % support the \includegraphics command and options

% \usepackage[parfill]{parskip} % Activate to begin paragraphs with an empty line rather than an indent

%%% PACKAGES
\usepackage{booktabs} % for much better looking tables
\usepackage{array} % for better arrays (eg matrices) in maths
\usepackage{paralist} % very flexible & customisable lists (eg. enumerate/itemize, etc.)
\usepackage{verbatim} % adds environment for commenting out blocks of text & for better verbatim
\usepackage{subfig} % make it possible to include more than one captioned figure/table in a single float

\usepackage{csquotes}
\usepackage{graphicx} % support the \includegraphics command and options
\usepackage{amssymb}
\usepackage{amsmath, amsfonts, bbm, dsfont}
\usepackage[all]{xy}
\usepackage{MnSymbol}
\usepackage{soul}

\usepackage{tikz-cd}
\usepackage{tikzit}
\input{tikz_style.tikzstyles}

% These packages are all incorporated in the memoir class to one degree or another...

%%% HEADERS & FOOTERS
\usepackage{fancyhdr} % This should be set AFTER setting up the page geometry
\pagestyle{fancy} % options: empty , plain , fancy
\renewcommand{\headrulewidth}{0pt} % customise the layout...
\lhead{}\chead{}\rhead{}
\lfoot{}\cfoot{\thepage}\rfoot{}

%%% SECTION TITLE APPEARANCE
\usepackage{sectsty}
\allsectionsfont{\sffamily\mdseries\upshape} % (See the fntguide.pdf for font help)
% (This matches ConTeXt defaults)

%%% ToC (table of contents) APPEARANCE
\usepackage[nottoc,notlof,notlot]{tocbibind} % Put the bibliography in the ToC
\usepackage[titles,subfigure]{tocloft} % Alter the style of the Table of Contents
\renewcommand{\cftsecfont}{\rmfamily\mdseries\upshape}
\renewcommand{\cftsecpagefont}{\rmfamily\mdseries\upshape} % No mathbf!

\usepackage[
backend=biber,
style=alphabetic,
giveninits=true,
doi=false,
url=true,
isbn=false]{biblatex}
\AtEveryBibitem{\clearfield{month}\clearlist{language}}

\renewbibmacro{in:}{}

\addbibresource{bibliography.bib}

\usepackage{amsthm}
\usepackage{tikz-cd}
\usepackage{appendix}
\usepackage{framed, color}
\usepackage[tikz]{bclogo}
\usepackage{mdframed}

\tikzcdset{row sep/normal= 0.8 cm,
column sep/normal= 0.8 cm}


\DeclareMathOperator{\flex}{\textbf{\textasciicircum}}

\newtheorem{theo}{Theorem}[section]
\newtheorem{theorem}[theo]{Théorème}
\newtheorem{prop}[theo]{Proposition}
\newtheorem{lemma}[theo]{Lemma}
\newtheorem{coro}[theo]{Corollary}
\newtheorem{hyp}[theo]{Hypothesis}
\newtheorem{claim}[theo]{Claim}

\newtheorem{recall}[theo]{Recall}

\theoremstyle{definition}
\newtheorem{fact}[theo]{Fact}
\newtheorem{defi}[theo]{Definition}
\newtheorem{definition}[theo]{Definition}
\newtheorem{notation}[theo]{Notation}

\newtheorem{rem}[theo]{Remark}
\newtheorem{remark}[theo]{Remark}

\newtheorem{ex}[theo]{Exemple}
\newtheorem{exemple}[theo]{Exemple}

\newtheorem{contre-ex}[theo]{Counter-example}






\newtheorem{motiv}[theo]{Motivation}
\newtheorem{conj}[theo]{Conjecture}
\newtheorem{conv}[theo]{Convention}
\newtheorem{constr}[theo]{Construction}
\newtheorem{convention}[theo]{Convention}
\newtheorem{construction}[theo]{Construction}


	

\newtheorem{exo}[theo]{Exercise}
\newtheorem{qst}[theo]{Question}
\newtheorem{checklist}[theo]{Checklist}


\usepackage{hyperref}
\hypersetup{
    colorlinks=true,
    linkcolor=black,
    filecolor=black,      
    urlcolor=black,
    citecolor=black,
}

\usepackage{tikz}
\usetikzlibrary{intersections}
\usepackage{quiver}

\newcommand{\brightarrow}{\mathrel{\vcenter{\hbox{\ooalign{$\rightarrow$\cr\hidewidth$|$\hidewidth}}}}}
 \newcommand{\bmrightarrow}{\mathrel{\vcenter{\hbox{\ooalign{$\rightarrow$\cr\hidewidth$\mid$\hidewidth}}}}}
  \newcommand{\bvrightarrow}{\mathrel{\vcenter{\hbox{\ooalign{$\rightarrow$\cr\hidewidth$\vert$\hidewidth}}}}}
    \newcommand{\bsmrightarrow}{\mathrel{\vcenter{\hbox{\ooalign{$\rightarrow$\cr\hidewidth$\shortmid$\hidewidth}}}}}









\newcommand{\vin}{\,\varepsilon\,}

\newcommand{\ON}{\mathbf{On}}
\newcommand{\NO}{\mathbf{No}}
\newcommand{\bP}{\mathbb{P}}
\renewcommand{\AA}{\mathcal{A}}
\newcommand{\BB}{\mathcal{B}}
\newcommand{\CC}{\mathcal{C}}
\newcommand{\DD}{\mathcal{D}}
\newcommand{\FF}{\mathcal{F}}
\newcommand{\II}{\mathcal{I}}

\DeclareMathOperator{\GG}{\mathcal{G}}
\DeclareMathOperator{\HH}{\mathcal{H}}

\newcommand{\cS}{\mathcal{S}}
\newcommand{\cT}{\mathcal{T}}

\newcommand{\M}{\mathrm{M}}
\newcommand{\N}{\mathrm{N}}
\newcommand{\T}{\mathrm{T}}

\DeclareMathOperator{\Aa}{\mathbb{A}}
\DeclareMathOperator{\Bb}{\mathbb{B}}
\DeclareMathOperator{\Dd}{\mathbb{D}}
\DeclareMathOperator{\Ee}{\mathbb{E}}
\DeclareMathOperator{\Gg}{\mathbb{G}}
\DeclareMathOperator{\Nn}{\mathbb{N}}
\DeclareMathOperator{\Qq}{\mathbb{Q}}
\DeclareMathOperator{\Cc}{\mathbb{C}}
\DeclareMathOperator{\Ff}{\mathbb{F}}
\DeclareMathOperator{\Rr}{\mathbb{R}}
\DeclareMathOperator{\Zz}{\mathbb{Z}}
\DeclareMathOperator{\Vv}{\mathbb{V}}
\newcommand{\Part}[1]{\mathcal{P}(#1)}
\newcommand{\lens}[4]{\left(\begin{array}{c}#1 \\#2  \\\end{array}\right) \leftrightarrows \left(\begin{array}{c}#3 \\#4  \\\end{array}\right)}
\newcommand{\chart}[4]{\left(\begin{array}{c}#1 \\#2  \\\end{array}\right) \rightrightarrows \left(\begin{array}{c}#3 \\#4  \\\end{array}\right)}

\DeclareMathOperator{\isom}{\simeq}
\DeclareMathOperator{\imp}{\Rightarrow}
\DeclareMathOperator{\sminus}{\backslash}
\renewcommand{\phi}{\varphi}
\DeclareMathOperator{\Aut}{Aut}






\title{Report on Work Completed}



\author{Paul $\mathrm{Wang}$}

\date{}












\begin{document}
	

	
	\setcounter{tocdepth}{2}
	\maketitle
	

	% \footnotetext[2]{\copyright \,  2023. This manuscript version is made available under the CC-BY-NC-ND 4.0 license http://creativecommons.org/licenses/by-nc-nd/4.0/}
	
	  \tableofcontents

\part{Geometric Model Theory}

First-order model theory\footnote{Recall that this means that the only quantifiers allowed are those ranging over elements; there is no quantification over subsets.} aims to study structures and theories\footnote{For this part, a first-order theory is a class of first-order structures defined by a collection of first-order axioms.} (algebraic, combinatorial, etc.) by relying on tools from (first-order) logic. In its modern version, the central objects are \emph{definable sets}, that is, collections of elements defined by formulas in the language. In classical logic, definable sets form Boolean algebras (intersection, complement, union); by Stone duality, one can therefore study the associated profinite topological spaces\footnote{That is, compact spaces whose opens are unions of clopen sets.}. These spaces are called \emph{type spaces}, and provide a more "geometric" viewpoint for the study of definable sets\footnote{Just as the study of affine schemes can be considered more "geometric" than the study of commutative rings; in fact, \emph{profinite spaces are affine schemes} where the sheaf of rings can be reconstructed from the topology, via clopen sets.}.


If the combinatorial complexity of definable sets is low, the theory is considered tame. For example, \emph{stable theories}, defined in \cite{Shelah-Thesis}, and studied in more detail in \cite{She-NIP}, are those where one cannot define an infinite total order.

\begin{exemple}
	\begin{itemize}
		\item In algebraically closed fields, equipped only with the field structure, the only definable sets are the \emph{constructible} ones, that is, Boolean combinations of zero loci of polynomials. In particular, we say that the theory \emph{eliminates quantifiers}: even allowing ourselves to use quantifiers, the only sets that can be defined by first-order formulas are already definable without quantifiers.
		\item Conversely, in the ring of integers, the complexity of quantification has no bound, and is related to subtle questions of computability: the sets definable by an existential formula are exactly those that are recursively enumerable, that is, for which there exists a computable function that enumerates them, etc.

	\end{itemize}
\end{exemple}





A general principle, \emph{Zilber's trichotomy}, states that the behavior of a tame theory depends on the type of definable algebraic structures\footnote{That is, whose underlying set and operations are definable.} that appear in it: either no infinite definable group, or infinite definable abelian groups but no infinite definable field, or infinite definable fields. These ideas have been formalized in various contexts, such as Zariski geometries\footnote{A framework aiming to capture in an abstract manner the geometric behavior of algebraic curves.} \cite{hru-zilber-zariski} or o-minimal theories\footnote{A viewpoint, originally model-theoretic, on real geometry and analysis, consisting of restricting classes of functions and sets in order to exclude pathological behaviors.} \cite{omin-trichotomy}. A related question, at the heart of my work, is that of the \emph{classification of definable and interpretable groups and fields}\footnote{An interpretable set is a quotient of a definable set by a definable equivalence relation.} in a given structure.


\section{Definable Covers and Groupoids}

When studying the fine structure of definable sets in a given theory or structure, the notion of \emph{cover} appears quite quickly. Given a structure $\mathbb{U}$, a \emph{cover of} $\mathbb{U}$ \emph{via a\footnote{We will focus here on the case of a single new sort; generalizing the notion to several sorts, or even infinitely many, poses no particular difficulty.} new sort} $S$ is a structure $\mathbb{U}'$, whose underlying set is the disjoint union of that of $\mathbb{U}$ and of $S$, such that the definable structure induced on $\mathbb{U}$ coincides with the original one. We also say that $\mathbb{U}$ is \emph{stably embedded} in $\mathbb{U}'$. An important fact, which explains the terminology, is that, in a stable structure, every definable set is stably embedded.


\subsection{Internal Covers and Binding Groupoids}

An interesting class of covers is that of \emph{internal covers}. In a given ambient structure, a definable set $X$ is \emph{internal} to a definable set $Y$ if there exists a definable bijection between $X$ and a definable set\footnote{We sometimes say "interpretable", because a definable quotient is involved; there is a canonical way to extend the definable structure to definable quotients, called \emph{imaginaries}.} $Z \subseteq Y^k/E$, where $k$ is an integer, and $E$ is a definable equivalence relation on $Y^k$.


A crucial point is that such a definable bijection may require additional parameters (to be defined), beyond those used to define $X$ and $Y$.

\begin{exemple}
	Consider the theory of differentially closed fields of characteristic zero\footnote{This is the theory of algebraically closed fields of characteristic zero equipped with a derivation, i.e. an additive endomorphism satisfying the Leibniz formula, such that every polynomial differential equation in one variable, non-degenerate, has "many" solutions.}, denoted $\mathrm{DCF}_0$. This theory is stable, and eliminates quantifiers.

	The field of constants, defined by the differential equation $x' = 0$, is algebraically closed, and its induced structure is reduced to field operations. Also consider the set $X$ defined by the equation $x'=1$. Then, by linearity, every element of $X$ provides a definable bijection between $X$ and the field of constants. However, without parameters, there are no interesting definable functions between the two.
\end{exemple}


This dependence on parameters is controlled by a binding \emph{groupoid}, introduced by Hrushovski in \cite{HRUSHOVSKI_2012}, which is an intrinsic version of \emph{binding groups}, studied in stability theory, notably by Zilber \cite{Zilber_1980} and Poizat \cite{Poizat_1983}, the latter developing a model-theoretic generalization of Galois theory of fields, including definable quotients. Recall that a groupoid is given by a collection of objects and invertible morphisms, with an associative and unital composition law; the main difference from the notion of group is that the composition of two morphisms is only defined if the domain of one equals the image of the other. The interest, here as in other areas\footnote{One can think for example of the difference between fundamental group and fundamental groupoid in algebraic topology, the first notion requiring the choice of a basepoint.}, is \emph{the absence of choice}.



\subsection{Questions}

Given a structure $\mathbb{U}$ and a set $A$ definable without parameters, a cover $(\mathbb{U}, S)$ of $\mathbb{U}$ is called \emph{1-analyzable over A} if there exists
a surjective function definable without parameters $f: S \rightarrow A$, whose fibers $S_a$ are internal to $\mathbb{U}$.

\begin{exemple}\label{ex_DCF_1-an}
	In the theory of differentially closed fields, the set $X$ defined by the differential equation ${(\frac{x'}{x})' = 0}$ and the condition $x \neq 0$, gives a $1$-analyzable cover over the field of constants $C$, via the logarithmic derivative function $x \in X \mapsto \frac{x'}{x} \in C$. Indeed, for every $\lambda \in C$, the set $X_{\lambda} = \lbrace x\neq 0 \, | \, x' = \lambda \cdot x \rbrace$ is internal to $C$: every solution defines, by multiplication, a bijection between $X_{\lambda}$ and $C\setminus \lbrace 0 \rbrace$.

	Note that $X$ is a multiplicative subgroup of the ambient differential field, and fits into the following short exact sequence of abelian groups: $1 \rightarrow C^{\times} \rightarrow X \rightarrow C \rightarrow 0$.
\end{exemple}




\begin{itemize}
	\item Can Hrushovski's constructions be extended to the case of $1$-analyzable covers?
	\item If so, can we give a criterion for a $1$-analyzable cover to be in fact internal?
	\item Are there uniform versions in families?

\end{itemize}




\subsection{Existing Work}

\begin{itemize}
	\item In \cite{HRUSHOVSKI_2012}, Hrushovski proves, among other things\footnote{He also studies connections with amalgamation questions, which go beyond the scope of this report.}, a correspondence, for any structure $\mathbb{U}$, between definable groupoids\footnote{With a small technical hypothesis, noted by Haykazyan and Moosa.} \emph{without parameters} in $\mathbb{U}$ and internal covers of $\mathbb{U}$ with one additional sort. The crucial idea is that the morphisms of the definable groupoid are definable bijections witnessing the internality of the additional sort.
	\item In \cite{Haykazyan_Moosa_2018}, Haykazyan and Moosa prove on the one hand that Hrushovski's correspondence comes from an equivalence of categories, for good notions of morphisms of internal covers\footnote{Specifically, these are functions between the additional sorts, definable in a common internal cover.} and of morphisms of definable groupoids. Faithful to the principle of not making arbitrary choices, they define a morphism of groupoids as, essentially, a definable \emph{profunctor}, that is, a category extending the disjoint union by adding morphisms between the objects of the two groupoids; they also require important homogeneity properties.

	Moreover, they extend the correspondence to the case of $1$-analyzable covers with independent fibers\footnote{A strong condition, expressing the absence of interaction, at the level of definable sets, between the fibers.}, that is, they give a "uniform version along definable families" of the construction. 

	\item In his thesis, Jimenez proves a uniform result \cite[Theorem 3.1.3]{Jimenez_thesis_2020} along families of internal covers that he calls \emph{relatively internal pairs}. He also constructs \emph{Delta groupoids}\footnote{A notion similar, but simpler, to that of \emph{simplicial} groupoids.}, which he uses to characterize internality for his relatively internal pairs, via a degeneracy condition. However, as explained in the discussion \cite[End of Section 3.2]{Jimenez_thesis_2020}, his constructions do not answer the question of non-canonicity of parameters.
		%In collaboration with Pillay and Jaoui, he develops the notion of \emph{uniform relative internality}, which is equivalent to the degeneracy of Delta groupoids

\end{itemize}




\subsection{Contributions}
In the article \cite{Wang_2021}, I define a notion of \emph{simplicial groupoid\footnote{Which should rather be called Delta groupoid, as in \cite{Jimenez_thesis_2020}.} definable} over a definable set $A$. Such an object is given by:

\begin{enumerate}
	\item A family $\mathcal{G}_n$, $n \geq 1$, of definable groupoids without parameters, where, for every $n$, the groupoid $\mathcal{G}_n$ admits a definable partition indexed by the interpretable set\footnote{Indeed, this set is isomorphic to the quotient of the set $A^{=n}$ of $n$-tuples of $A$ without repetition, by the action of the symmetric group $\mathfrak{S}_n$.} of subsets of $A$ of cardinality $n$.
	\item Morphisms $\iota_{n, m}: \mathcal{G}_n \rightarrow \mathcal{G}_m$, for $n < m$, between definable groupoids, in the sense of Haykazyan and Moosa \cite{Haykazyan_Moosa_2018}, that is, given by a definable extension of the definable category $\mathcal{G}_n \sqcup \mathcal{G}_m$, i.e. collections of morphisms from objects of $\mathcal{G}_n$ to those of $\mathcal{G}_m$.


	I require that these groupoid morphisms $\iota_{n, m}$ be compatible with the partitions of $\mathcal{G}_n$ and $\mathcal{G}_m$, that the functions inside be \emph{injective}, and that the equality $\iota_{n,m} \circ \iota_{k, n} = \iota_{k, m}$ hold for $k < n <m$, in the sense of composition of groupoid morphisms (see \cite{Haykazyan_Moosa_2018}).
\end{enumerate}


I then define a notion of \emph{morphism between simplicial groupoids over A}.



I then prove the following results, for any $\mathbb{U}$ and any $A$:

\begin{itemize}
	\item There exists a correspondence between $1$-analyzable covers over $A$, uniformly finitely generated\footnote{Technical finiteness condition.}, locally stably embedded\footnote{Condition on intermediate internal covers.}, and simplicial groupoids over $A$ satisfying the disjoint union property\footnote{Condition expressing the fact that objects at high degree are disjoint unions of objects at low degree.}.
	\item This correspondence comes from an equivalence of categories, for categories of $1$-analyzable covers and simplicial groupoids \emph{consisting of isomorphisms}.
	\item The constructions generalize by removing the finiteness condition on the covers side; the resulting simplicial groupoids are then only pro-definable, i.e. obtained by possibly infinite products of definable sets.
	\item I then compute the corresponding objects for example \ref{ex_DCF_1-an}, as well as for the short exact sequence of abelian groups $0 \rightarrow {(\mathbb{Z}/ 2 \mathbb{Z})}^{\mathbb{N}} \rightarrow {(\mathbb{Z}/ 4 \mathbb{Z})}^{\mathbb{N}} \rightarrow {(\mathbb{Z}/ 2 \mathbb{Z})}^{\mathbb{N}} \rightarrow 0$.

	I show, for these two examples, that the simplicial groupoids restricted to degrees $\leq 3$ suffice to capture the total structure of the covers: only the group law matters, and it is already encoded in degrees $\leq 3$.
\end{itemize}

\begin{mdframed}{\textbf{Ideas for constructing simplicial groupoids:}}
	\begin{itemize}
		\item Following Hrushovski's suggestion in \cite{HRUSHOVSKI_2012}, given a $1$-analyzable cover $(\mathbb{U}, S)$ over a definable set $A$, I consider, for any finite subset $\overline{c} \subset A$, the \emph{internal} cover defined by the union of fibers $S_{\overline{c}} = \bigcup\limits_{a \in \overline{c}} S_a$. Hrushovski's construction then applies, I obtain a definable groupoid with parameters $\overline{c}$, which I denote $\mathcal{G}_{\overline{c}}$.
		\item I show that, for every $n$, the above construction can be done uniformly in $\overline{c} \subset A$ of cardinality $n$; I thus obtain definable groupoids $\mathcal{G}_n$.
		\item The morphisms $\iota_{n, m}: \mathcal{G}_n \rightarrow \mathcal{G}_m$ are constructed from the inclusions $S_{\overline{c}} \hookrightarrow S_{\overline{c}'}$, for $\overline{c} \subset \overline{c}' \subset A$ finite.
	\end{itemize}

This construction is close to that of Jimenez; the main difference is that my groupoid morphisms are profunctors, and not simply functors. This was already a crucial point in the work of Haykazyan and Moosa, which made their constructions work.

\end{mdframed}


\begin{mdframed}{\textbf{Ideas for constructing 1-analyzable covers:}}
	Given a definable simplicial groupoid $\mathcal{G}$ over an $A$, I construct a $1$-analyzable cover over $A$ as follows:

	\begin{itemize}
		\item I construct by transfinite induction a \emph{coherent system of inclusions}, which corresponds to choosing one object per connected component at each degree, as well as a coherent family of injections among the functions of the $\iota_{n, m}$, between these objects. 
		\item Given a coherent system of inclusions, I construct an additional sort $S$, consisting of gluing copies of the chosen objects along the injections of the family. By construction, the sort $S$ comes with a definable surjection $S \rightarrow A$, whose fibers are internal to $\mathbb{U}$.
		\item To show that this does not depend on the choices made, I construct by induction, given two coherent systems of inclusions, a "coherent family of partial isomorphisms" between them, consisting of morphisms of $\mathcal{G}$. Such a family then assembles into an isomorphism of structures between the constructed extensions.
		\item To show that the structures I construct are indeed covers, I apply an automorphism extension criterion \cite[Appendix, Lemma 1]{ChaHru-ACFA}, and use again a transfinite induction construction.
	\end{itemize}


	This construction generalizes that of Haykazyan and Moosa \cite{Haykazyan_Moosa_2018}, with additional compatibility conditions, automatic for them given their hypothesis of independence of fibers.
\end{mdframed}



\section{Generically Stable Group Configurations}

The principle of the group configuration theorem is to \emph{detect the presence of a definable group} \emph{from combinatorial data}. This result belongs to the field of \emph{geometric model theory}, which, we recall, consists in studying first-order theories through the definable algebraic structures that appear in them. 



\subsection{Context}

A group configuration is the data of $6$ elements satisfying very specific independence and dependence properties; the essential example is the following: let $G$ be a definable group in a stable theory\footnote{This means that no definable relation induces an infinite total ordering relation; this is a strong hypothesis of combinatorial tameness.}, for example the theory of algebraically closed fields, and let $g_1$, $g_2$, $g_3$ be \emph{generic} elements of $G$, forming an \emph{independent family}\footnote{For example, if $G = \mathrm{GL}_n(\mathbb{C})$, the conjunction of these two conditions corresponds to requiring that the $g_i$ be in general position; the set of triples of $\mathrm{GL}_n(\mathbb{C})$ not satisfying this property is negligible for the Lebesgue measure restricted to ${\mathrm{GL}_n(\mathbb{C})} \times {\mathrm{GL}_n(\mathbb{C})} \times {\mathrm{GL}_n(\mathbb{C})} $.}. Then, the diagram below is a group configuration: 



\begin{center}
\begin{tikzpicture}
\coordinate (b3) at (0,0) ;
\coordinate (b2) at (0,-1) ;
\coordinate (b1) at (0,-2) ;
\coordinate (a2) at (1,-0.5) ;
\coordinate (a1) at (2,-1) ;
\coordinate (a3) at (1,-1) ;


\draw (b3) -- (a1);
\draw (b3) -- (b1);
\draw (b2) -- (a1);
\draw (b1) -- (a2);

%\path [name intersections={of=CD1 and BE1, name=(f1)}];

\draw (b3) node [above] {\(g_2 \cdot g_1\)};
\draw (b2) node [left] {\(g_{2}\)};
\draw (b1) node [left] {\(g_{1}\)};
\draw (a2) node [right] {\(g_3\)};
\draw (a1) node [right] {\(g_1 \cdot g_3\)};
\draw (a3) node [below] {\(g_2 \cdot g_1 \cdot g_3\)};

\end{tikzpicture}
\end{center}

Indeed, the two properties defining the notion of \emph{regular} group configuration\footnote{There is indeed a definition, and a theorem, for more general \emph{group actions} than an action by translation of a group on itself; we will ignore it here, despite its importance.} are the following:

\begin{itemize}
	\item In the diagram, any triple of non-collinear points forms an independent family\footnote{Every stable theory comes with a canonical notion of independence; part of the technical difficulty is to work with generalizations that make sense for unstable theories.}.
	\item For any line in the diagram, each point is algebraic over the other two\footnote{Which means that it is \emph{of finite orbit} under the action of automorphisms of the ambient structure fixing the other two elements; equivalently, there exists a formula having a finite number of solutions, among which the point in question, using as parameters the other two elements.}.
\end{itemize}



The statement of the group configuration theorem for stable theories, attributed to Hrushovski, is the following: 

\begin{theorem}
	For any (regular) group configuration in a stable theory, there exists a definable group\footnote{Technically, \emph{type-definable}, a slightly more general variant, where one allows infinite intersections of definable sets.} $\Gamma$ and three elements $g_1$, $g_2$, $g_3$ of the group $\Gamma$, generic and independent, such that the initial group configuration and that of $\Gamma$ constructed from the $g_i$ are equivalent, that is, interalgebraic point by point.


	Moreover, \emph{such a group is essentially unique}: given two equivalent group configurations constructed from independent generic elements for two definable groups $\Gamma$ and $\Gamma'$, there exists a definable virtual isogeny\footnote{In other words, a normal subgroup of $\Gamma \times \Gamma'$ for which the projections to $\Gamma$ and $\Gamma'$ have finite fibers, and images of finite index.} between $\Gamma$ and $\Gamma'$.

\end{theorem}


In other words, the examples presented above are essentially the only ones: in a stable theory, any sextuple of elements satisfying the independence and algebraicity properties above comes from independent generics of a definable group. This result has remarkable consequences: for example, the existence of locally modular regular types implies the existence of infinite type-definable abelian groups with regular generics. In other words, Hrushovski used a group configuration theorem to \emph{prove the existence of groups with certain properties}.


\subsection{Existing Work}
The only variant of the group configuration theorem proved, until then, outside the framework of stable theories, by Ben Yaacov-Tomasic-Wagner \cite{group-config-simple}, relies on a weaker hypothesis of \emph{simplicity} of the ambient theory, at the cost of a weaker conclusion as well: the obtained group is only "almost hyper-definable".

\subsection{Contributions}
In the published article \cite{Wang_group_config}, I prove a generalization of the original group configuration theorem, assuming only that the sextuple considered defines a \emph{generically stable type}, that is, that sequences of independent copies of this sextuple behave as if they were in a stable theory. The statement is essentially the same, except that I give more precise control over the parameters/elements used in the construction, which can be useful for applications of the theorem. The structure of the proof is similar.

At the technical level, the essential work consisted in using precisely the properties of generically stable types, in particular the notion of independence provided by abstract model theory, which behaves \emph{almost}\footnote{The technical subtlety that required the most effort is that one does not know in general (see \cite[Question 4.1]{Conant_Gannon_Hanson_2025}, which gives a positive answer for $\mathrm{NTP}_2$ theories) whether the concatenation of two independent copies of a generically stable type, also called \emph{tensor product} or \emph{Morley product}, is itself generically stable.} as well as in the case of stable theories.



As for applications, in view of the results obtained in stable contexts using the original group configuration theorem, this generalization is a step toward the development of analogous results relying on generically stable types.







\section{Interpretable Groups and Fields in Field Theories}

In \cite{Wang_interp_groups_2025}, I study the structure of definable/interpretable groups and fields, both in abstract frameworks and for explicit examples of valued field theories.



\subsection{Questions}
Given a field theory (in first-order logic) $T$:
\begin{itemize}
	\item What can we say about definable groups in $T$? Can we compare them definably to algebraic groups (in coordinates\footnote{"Algebraic groups in coordinates" are always definable, because "varieties in coordinates" and rational functions are, since field operations are part of the structure.})?
	\item What about definable fields in $T$? Are they reduced to finite extensions\footnote{Which are always definable, by choosing bases, in the sense of linear algebra.} of the ambient field?
	\item What can we say if we generalize these questions to the case of \emph{interpretable} groups and fields, i.e. if we allow quotients of definable sets by definable equivalence relations?
\end{itemize}

\subsection{(Valued) Field Theories}

For what follows, a valued field is given by a field $K$, an ordered abelian group\footnote{One requires that the order be total, and compatible with the group law.} $\Gamma$, which we will write multiplicatively here, and a surjective group homomorphism $|\cdot |: K^{\times} \rightarrow \Gamma$, called norm, satisfying the following ultrametric inequality: $|x+y| \leq \max (|x|, |y|)$. Usually, the norm is extended by adding an absorbing element to $\Gamma$, equal to the norm of the element $0$. We call $\Gamma$ the value group.

\begin{exemple} The following structures are valued fields:
	\begin{itemize}
		\item 	For any prime number $p$, and any real $r$ such that $0 < r < 1$, the field of rational numbers $\mathbb{Q}$, equipped with the $p$-adic norm $x \mapsto r^{v_p(x)}$, where $v_p(x)$ denotes the $p$-adic valuation of $x$. The value group is $r^{\mathbb{Z}}$, isomorphic to $\mathbb{Z}$. Up to isomorphism, this structure does not depend on the choice of $r$.
		\item For any prime number $p$, and any real $r$ with $0 < r <1$, the completion of $\mathbb{Q}$ according to the $p$-adic norm, denoted $\mathbb{Q}_p$. It is called the field of $p$-adic numbers, and, up to isomorphism, it does not depend on the choice of $r$. 
		\item For any field $K$, the field of rational fractions $K(t)$, with a norm associated with the $t$-adic valuation.
		\item For any field $K$, the field of Laurent series $K((t))$, with a $t$-adic norm, which is the completion of $K(t)$.
		\item For any field $K$ and any ordered abelian group $\Gamma$, the field of Hahn series $K(X^{\Gamma})$, whose elements are functions from $\Gamma$ to $K$ with co-well-ordered support (i.e. well-ordered for the dual order of $\Gamma$); addition is defined termwise, and multiplication is computed by convolution.

	\end{itemize}

\end{exemple}



In a valued field, the ultrametric inequality implies that the closed unit ball $\mathcal{O}= \lbrace x \, : \, |x| \leq 1 \rbrace$ is a local ring, whose maximal ideal $\mathfrak{m}$ is the open ball, $\mathfrak{m}= \lbrace x \, : \, |x| < 1 \rbrace$. The quotient $\mathcal{O} / \mathfrak{m}$ is called the \emph{residue field}.


\begin{mdframed}{\textbf{Definition:}}
	A (valued) field theory is a theory with a marked definable (valued) field\footnote{Just as a pointed topological space is a topological space with a marked point.}.
\end{mdframed}


An important point is that the definable (valued) field in question may have additional structure; the challenge is to understand how to study these cases without necessarily going through a detailed analysis of said additional structure. 

\subsection{Existing Work}
\begin{itemize}
	\item The first result, due to Hrushovski, inspired by the ideas of \textcite{Wei-GpCh}, states that every definable group\footnote{In fact, "definable" and "interpretable" are equivalent in algebraically closed fields: we say that these \emph{eliminate imaginaries}.} in a pure algebraically closed field is definably isomorphic to an algebraic group (in coordinates). This uses a "\emph{group chunk}" theorem, which is a way of reconstructing a definable group from generic data, similar to the group configuration theorem.
	\item Hrushovski's result was used by Poizat \cite{Poi-GenNS}, to show that every definable field in a pure algebraically closed field is definably isomorphic to it. Also relying on these ideas, it was proved in \cite[Proposition 2.5]{Pil-OMinGp}, that every definable group in an o-minimal structure is naturally a topological group, that every definable field in this context is either real closed or algebraically closed, and that, in the case of the pure field of real numbers, definable groups are Lie groups. These results were then extended in \cite[Theorems A and C]{HruPil-GpPFF}, to show that every Nash group over a real or p-adic field is locally isomorphic to the set of points of an algebraic group, and that every definable group in a pseudo-finite field\footnote{That is, an infinite field satisfying all first-order properties common to finite fields.} is virtually isogenous to the set of rational points of an algebraic group. Regarding differentially closed fields\footnote{Which are fields of characteristic zero equipped with a derivation, and for which algebraic differential equations have "sufficiently many solutions".}, Pillay \textcite[Corollary 4.2]{Pillay1997SomeFQ} proved that every connected definable group embeds into an algebraic group, and Suer \textcite[Theorem 3.36]{Suer2007ModelTO} showed that the only infinite definable fields, up to definable isomorphism, are the differential field and the field of constants.
	
	\item The most general theorem so far is \cite[Theorem 2.19]{Stabilizers-NTP2}. Recall that a discrete group is \emph{amenable} if there exists a finitely additive probability measure, left-invariant, defined on all subsets. Then, a definable group, in a certain theory, is called \emph{definably amenable} if it admits a finitely additive probability measure, left-invariant, defined on definable sets. Theorem 2.19 of Montenegro, Onshuus and Simon treats definably amenable groups in all perfect field theories having reasonable model-theoretic properties\footnote{Specifically, which are $\mathrm{NTP_2}$ and where the algebraic closures in the sense of model theory and in the sense of fields coincide.}. 
	
	\item In the more specific context of valued fields\footnote{Ultrametric, not necessarily of rank one.}, under certain technical hypotheses, the classification of interpretable fields was carried out in \cite[Theorem 7.1]{fields_interp_in_various_val_fields}: the only interpretable fields are finite extensions of the ambient field or of the residue field. Other results include work done in \cite{ppp_def_groups_codf}, which prove that definable groups in various differential fields can be definably embedded in algebraic groups, and \cite[Theorem 1]{semisimple_groups_interp_in_various_val_fields}, showing that definably semi-simple interpretable groups in enrichments of algebraically closed, real closed or p-adically closed valued fields, are virtually isogenous to products of linear groups over the valued field and over the residue field. In the specific case of pure algebraically closed valued fields, the classification of interpretable fields was initially proved in \cite{HruRK-MetaGp}, using a notion called metastability, which focuses on generically stable types. Finally, in the framework of 1-h-minimality, a notion aiming to capture the idea of tame geometry in non-Archimedean contexts, it is shown in \cite{AcoHass-1hmin} that, as in the real case, definable groups admit a Lie structure, and that the only definable fields are finite extensions of the ambient field. 


\end{itemize}

\subsection{Contributions}

%The work of Halevi, Hasson and Peterzil treats interpretable groups without requiring complete elimination of imaginaries, a remarkable characteristic. They are also quite precise, since the group homomorphisms they define are virtual isogenies, that is, close to being isomorphisms. The main limitation, so to speak, is that their tools require fairly strong hypotheses, and thus only cover certain cases. Similarly, the geometric approach of Acosta López and Hasson produces precise results, for example excluding definable subfields of the ambient field. On the other hand, Peterzil, Pillay and Point construct definable embeddings of definable groups into algebraic groups, whose images can be small, for large classes of enriched differential fields. However, they do not treat imaginaries, just like Montenegro, Onshuus and Simon.

\begin{itemize}
	\item My work is similar to \cite[Theorem 2.19]{Stabilizers-NTP2}: I first prove a purely abstract theorem on constructing definable group homomorphisms \cite[Theorem 3.1.47]{Wang_thesis_2025}, involving two theories $T_0$ and $T_1$, where $T_0$ is superstable\footnote{A notion of tameness slightly stronger than stability.}, $T_1$ is NIP\footnote{A notion of tameness weaker than stability, covering o-minimal theories, notably the field of reals, many valued field theories, including p-adic fields, etc.} and $T_0$ is a "restriction"\footnote{In a precise technical sense.} of $T_1$. 
	
	
	A key notion is that of a definable set \(X\) that \emph{does not see} another definable set \(Y\): this essentially means that all definable functions \(X \rightarrow Y\) have finite image. The result is then the following: \emph{every definably amenable group in the NIP theory admits a definable group homomorphism to a group of the superstable theory, whose kernel does not see the definable sets of the superstable theory.}
	
	\item Then, by applying this abstract theorem for $T_0$ the theory of algebraically closed fields, I obtain a very general result \cite[Theorem 3.2.2]{Wang_thesis_2025} for \emph{algebraically bounded}\footnote{That is, where the algebraic closures in the sense of model theory and field theory coincide.} field theories, and I can cover more cases than \cite[Theorem 2.19]{Stabilizers-NTP2}: under moderate hypotheses, \emph{all \emph{interpretable}\footnote{Theorem \cite[Theorem 2.19]{Stabilizers-NTP2} only treats the case of \emph{definable} groups and finite kernels.} definably amenable groups admit a definable morphism to an algebraic group over the ambient field, whose kernel does not see the field.} In other words, these definably amenable interpretable groups admit short exact sequences, where the right term is a subgroup of an algebraic group, and the left term does not see the field.
	\item Similarly, taking for $T_0$ the theory of differentially closed fields, I prove analogous results \cite[Theorem 3.3.2]{Wang_thesis_2025} for interpretable groups in algebraically bounded differential field theories\footnote{The analogue of algebraically bounded field theories taking into account the presence of the derivation.}.
	\item As for interpretable fields, I prove \cite[Corollary 3.2.19]{Wang_thesis_2025}, under moderate hypotheses, the following dichotomy: \emph{an infinite interpretable field admits a definable embedding into a finite extension of the ambient field, or does not see the latter, in which case we call it \emph{purely imaginary\footnote{Which is generally the case for the residue field of a Henselian valued field, for example.}}}. 
	
	This relies on the intermediate result \cite[Proposition 3.2.7]{Wang_thesis_2025}: given two infinite definable fields $F$ and $K$, \emph{in an arbitrary theory}, if the group of affine transformations of $F$ embeds definably into an algebraic group over $K$, then the field $F$ embeds definably into a finite extension of $K$. 
	\item I then treat specific examples: general classes of Henselian valued fields \cite[Theorems 3.3.32 and 3.3.33]{Wang_thesis_2025} covering examples already known in the literature, and differentially closed valued fields\footnote{These are fields of characteristic zero equipped with a valuation and a derivation that "do not interact": all configurations permitted by algebra are realized, in particular non-trivial algebraic differential equations have solution sets dense for the valuation topology.}. For the latter, I prove \cite[Theorem 3.3.20]{Wang_thesis_2025} that the only infinite interpretable fields are the valued field, its field of constants, and the residue field.

\end{itemize}



These results encompass in a general framework the vast majority of examples previously studied in the literature, along with their proofs; the fact that they concern \emph{interpretable} groups allows applications to the study of imaginaries, i.e. definable quotients. Moreover, the most abstract theorems \cite[Theorem 3.1.47]{Wang_thesis_2025} \cite[Theorem 3.1.30]{Wang_thesis_2025} can apply to frameworks other than field theories.

%\section{Contribution included in the thesis: Classification of definable rings in algebraically closed fields}





\section{D-Henselian Valued Fields and Generically Stable Types}

In differentially closed valued fields, the derivation has generic behavior with respect to the valuation: all non-trivial differential equations possess dense sets of solutions. Although this hypothesis facilitates study by model-theoretic methods, one may want to consider differential valued fields where the derivation is continuous for the valuation topology, a more satisfying behavior from a geometric viewpoint. As shown in \cite{Sca-DValF}, there exists a whole family of interesting complete first-order theories for differential valued fields of equicharacteristic $0$, with $1$-Lipschitz derivations; these fields are sometimes called $D$-Henselian. I focused on the most generic case, namely where the residue field, with induced derivation, is a differentially closed field; I denote this theory $VDF$. Examples of structures satisfying these axioms are given by Hahn series fields $K((t^{\Gamma}))$, where $\Gamma$ is a divisible ordered abelian group, the residue field $K$ is a differentially closed field, and the derivation is constructed from that of $K$ by differentiating the coefficients termwise. 






\subsection{Existing Work}
\begin{itemize}
	\item In \cite{Sca-DValF}, Scanlon gives an axiomatization of $VDF$, shows that it is complete, and proves a quantifier elimination result, i.e. gives a description of the structure of definable sets; in fact, he proves these results for a general class of valued field theories with $1$-Lipschitz derivation.
	\item In \cite{Rid-VDF}, Rideau-Kikuchi proves finer results: description (in a certain sense) of imaginaries, i.e. quotients of definable sets by definable equivalence relations, density of definable types\footnote{Which have good properties, notably the existence of canonical extensions when enlarging the base field; every generically stable type is definable.} in type spaces, and metastability, which roughly corresponds to the presence of many generically stable types\footnote{Which we recall have very good regularity properties, from a model-theoretic viewpoint.}.
\end{itemize}

Concerning generically stable types in algebraically closed valued fields, and metastability, the work of Hrushovski and Loeser \cite{HruLoe} shows the richness of the notion: they construct model-theoretic analogues of Berkovich spaces, whose points are generically stable types\footnote{In fact stably dominated, which is equivalent here.}, and then construct comparisons with usual Berkovich spaces, which allows them to prove results about the latter, notably the existence of deformation retractions onto simplicial complexes, at a level of generality greater than what was known before.



\subsection{Questions}

The questions I consider are the same as for \cite{Wang_interp_groups_2025}, namely: 
\begin{itemize}
	\item What can we say about definable groups in the theory $VDF$? Do those that do not use imaginaries embed into algebraic groups?
	\item What about definable fields?
\end{itemize}



\subsection{Contributions}
In the last chapter of my thesis \cite[Chapter 4]{Wang_thesis_2025}, I study the above questions. A crucial point is that the results of the previous section do not apply for the theory $VDF$: a major obstacle is that the algebraic closure in $VDF$ is possibly larger than that induced by the theory of differentially closed fields: due to the $1$-Lipschitz condition, certain differential equations have a unique solution\footnote{For reasons similar to Picard's fixed point theorem, and uniqueness results for solutions of ordinary differential equations.}! An example is given in \cite[Proposition 3.1]{Rid-VDF}. 

However, a weaker property is satisfied: every element of the valued field that is algebraic\footnote{In the sense of model theory, i.e. which satisfies a formula having a finite number of solutions.} over a set of parameters is \emph{differentially algebraic} over it, that is, it is of finite rank in the sense of $DCF$, the theory of differentially closed fields. Thus, by working "up to finite rank data", I carry out constructions similar to those that precede. The idea is to follow the variant of Hrushovski's group configuration theorem with regular types and weights\footnote{For details, see for example \cite{Bouscaren1989TheGC}}. However, due to technical difficulties\footnote{Namely, because this variant sometimes involves taking deviant extensions of the types in play.}, I use an additional hypothesis on the generic of the group: I require that it be generically stable and \emph{orthogonal to all types of finite rank}\footnote{This corresponds to \emph{stability under base change along} (differential valued field) \emph{extensions that are differentially algebraic}. The typical example is the generic point/type of the valuation ring, whose image in the residue field is the differentially transcendental point/type.}. Thus, I prove the following results:

\begin{itemize}
	\item If $G$ is a definable group, living in the valued field, that admits a generically stable generic point/type orthogonal to all types of finite rank, then it admits a definable morphism\footnote{Defined at least on its connected component.} to an algebraic group, whose kernel is of finite rank. See \cite[Theorem 4.1.32]{Wang_thesis_2025}.
	\item If $F$ is a definable field, living in the valued field, that possesses a definable subring $R$ of the same non-zero differential transcendence degree, where $R$ admits a generic, additive and multiplicative, generically stable and orthogonal to types of finite rank, then $F$ is definably isomorphic to the ambient differential valued field. See \cite[Theorem 4.1.37]{Wang_thesis_2025}.
\end{itemize}



\emph{These results are, to my knowledge, the first obtained in a framework where the usual hypotheses on algebraic closure are not satisfied}; they are thus a first step toward a finer study of structures of this kind. For example, although the proofs do not properly use the metastability of the theory $VDF$, the good properties of generically stable types are crucial, which confirms, if confirmation were needed, the importance of the latter. 


Another class of interesting structures where these ideas could be adapted is that of valued fields with automorphisms\footnote{Which are required to be isometries \cite{AzgvdD}, or to be contracting \cite{Dor_Halevi_w_Kaplan_2025}, for the valuation.}, where, as for $VDF$, certain equations involving the automorphism have a unique solution.


\section{Residual Domination in Henselian Valued Fields}

The notion of \emph{stable domination}, briefly mentioned above, was introduced by Haskell, Hrushovski and Macpherson \cite{hhm}, to study algebraically closed valued fields. 
The starting point is the following observation: stable structures and theories have very good model-theoretic regularity properties; however, many interesting theories and structures not being stable, one cannot apply stability theory as such. 
The idea is then to consider the \emph{stable part} of a given structure, which is the "largest stable substructure", and to try to control behaviors in the total structure via their trace on the stable part. 



For example, in algebraically closed valued fields, the stable part is, up to internal cover, the residue field, which is a \emph{purely stably embedded} algebraically closed field\footnote{This means that the structure induced by the valued field is exactly the field structure, and nothing more.}. One then understands that only certain types, certain extensions, can be well controlled by the stable part: in a valued field, \emph{ramified} extensions (which enlarge the value group) or \emph{immediate} extensions (which enlarge neither the residue field nor the value group) have little chance of being understood via the residue field.


We say\footnote{Up to certain technical details, which we will ignore here.} that a type\footnote{That is, a point in the spectrum of an algebra of definable sets, i.e. a maximal coherent collection of definable sets.} is \emph{stably dominated} if \emph{the behavior of its generic extensions is controlled by what they induce on the stable part}. The canonical example, for algebraically closed valued fields, is that of the type of a \emph{purely residual monogenic extension of a spherically complete field}\footnote{This is a strong analogue of completeness in the general framework of ultrametric valued fields, where one requires that every chain of balls has an accumulation point; this notion is relevant when the valuation is not discrete.}. The notion was then generalized to that of \emph{residual domination}, in various contexts \cite{EHM19} \cite{ResDom2} \cite{Vic22} \cite{KRV24}.

\subsection{Existing Work}


\begin{itemize}
	\item The work of Haskell, Hrushovski and Macpherson \cite{hhm} contains fundamental results on stable domination, among others, base change properties for stably dominated types and metastability, relative to the value group, of the theory of algebraically closed valued fields. A particularly important result for what follows is the following: 
	

	In any theory, if $G$ is a definable group admitting a stably dominated generic $p$, then there exists a family of groups definable in the stable part $\mathfrak{g}_i$, and definable morphisms $f_i: G \rightarrow g_i$, such that the type $p$ is stably dominated by the family\footnote{Which means that the control described above only needs the information contained in these functions.} $(f_i)$.


	
	\item More recently, Cubides Kovacsics, Rideau-Kikuchi and Vicaría \cite{Kovacsics_Rideau-Kikuchi_Vicaria_2025} have defined a more general notion of \emph{residual domination} and shown the equivalence, under fairly weak technical hypotheses, between this notion and stable domination in the sense of algebraically closed valued fields, including for imaginaries\footnote{That is, equivalence classes of tuples from the valued field modulo definable equivalence relation; tuples in the residue field are a relatively simple example of imaginaries in valued fields.}. This is the most complete result to date.




\end{itemize}




\subsection{Main Results}
In \cite{Mutlu_Wang_arxiv_2025}, we prove the following results for Henselian valued fields of equicharacteristic zero:

\begin{itemize}
	\item Residual domination, for regular extensions of finite type, is characterized by the conjunction of the separated base property\footnote{Which means that there exist bases for which the norms of linear combinations are computed explicitly in terms of the norms of the coefficients.} and the absence of ramification. See \cite[Corollary ???]{Mutlu_Wang_arxiv_2025}.
	\item Residual domination in the sense of the ambient Henselian valued field is equivalent to stable domination in the sense of its algebraic closure\footnote{Equipped with the unique valuation extending its own, by henselianity; this structure is a "restriction" of the ambient Henselian field, because only formulas without quantifiers are taken into account.}, that is, in the sense of the theory of algebraically closed valued fields. See \cite[???]{Mutlu_Wang_arxiv_2025}. This result, less general than that of \cite{Kovacsics_Rideau-Kikuchi_Vicaria_2025}, was obtained independently, with more algebraic and simpler methods. It is useful to us for what follows.
	\item Suppose the theory considered is $\mathrm{NTP}_2$ (this is a fairly weak notion of tameness, covering a large number of examples). If $G$ is a definable group, living in the valued field, possessing a residually dominated generic, then there exists an algebraic group over the residue field $\mathfrak{g}$ and a surjective definable group homomorphism $f$ from\footnote{the connected component of} $G$ to $\mathfrak{g}$, such that all generics of\footnote{the connected component of} $G$ are dominated by $f$. In particular, all generics of $G$ are residually dominated. See ???
	\item Now suppose the theory considered is NIP\footnote{A more restrictive notion than $\mathrm{NTP}_2$, but still quite broad.}, and let $G$ be a definable group having a residually dominated generic, as before. Then, there exists a group $G_1$, definable and stably dominated in the sense of algebraically closed valued fields, and a definable group homomorphism $\iota$, with finite kernel, going from\footnote{the connected component of} $G$ into $G_1$, such that the image of every generic of\footnote{the connected component of} $G$ is generic in $G_1$ in the sense of algebraically closed valued fields.
	
	Moreover, the group $G_1$ and the morphism $\iota$ are essentially universal, and essentially functorial with respect to surjective morphisms of definable groups. See ???
\end{itemize}



\subsection{Personal Contributions}

For this article, a good part of the ideas were the result of interactions with my co-author Dicle Mutlu; the contributions that can be attributed to me are the following:

\begin{itemize}
    \item By exploiting the fine properties of genericity in the $\mathrm{NTP}_2$ framework, I prove that every generic of a group having \emph{one} residually dominated generic is itself residually dominated.

	\item I use the stabilizer theorem \cite[Theorem 2.15]{Stabilizers-NTP2} to construct definable group homomorphisms from generic data, which allows me to prove the universality and functoriality of the constructions of ???
	
\end{itemize}







\part{Categorical Systems Theory}

Categorical systems theory\footnote{See for example the introduction of \cite{Libkind_Myers_2025}.} aims to provide a unified theoretical framework for the study of dynamical systems, with technical aspects relying on notions from category theory. Several directions have emerged from this research field over time; the two most important, concerning my research, are the approach based on \emph{symmetric monoidal categories}, possibly with additional structure (see \ref{subs_smc}), and the \emph{coalgebraic viewpoint} on systems (see \ref{subs_coalg}). Indeed, my work falls within the new branch called \emph{doubly categorical theory}, or \emph{doubly operadic}, of systems \cite{Myers_2021} \cite{Libkind_Myers_2025}, which aims to combine the two aforementioned approaches in a single framework; it also relies on synthetic probability theory. It seems useful to me to begin by recalling what the "symmetric monoidal" and "coalgebraic" viewpoints consist of, as well as synthetic probability theory, before explaining the issues and questions in doubly categorical systems theory, and my contributions.

\section{The Double Categorical point of view}

\subsection{Symmetric Monoidal Approach}\label{subs_smc}
A symmetric monoidal category is a structure containing objects, morphisms between these objects, a product operation on objects (equipped with canonical symmetries), and sequential and parallel composition operations on morphisms, satisfying reasonable algebraic properties (associativity, etc.). The interest for representing dynamical systems is the following: if one can represent spaces of possible inputs or outputs as objects, systems as morphisms, and one indeed has parallel and sequential composition operations for systems, then \emph{the theory of symmetric monoidal categories automatically provides syntax and reasoning tools}. The most emblematic are \emph{string diagrams}\footnote{Called \emph{string diagrams} in English; see for example the nLab page on the subject: \url{ncatlab.org/nlab/show/string+diagram}.}, which allow rigorous calculations based on graphical manipulations of diagrams. 


One of the most striking applications is the \emph{ZX-calculus} \cite{Danos_Kashefi_Panangaden_2007}, used to represent quantum computations based on qubits, whose pedagogical virtues, presumably stemming from the graphical approach and the relative simplicity of the language, have been experimentally tested \cite{Coecke_Kissinger_Gogioso_Dündar-Coecke_Puca_Yeh_Waseem_Pothos_Pfaendler_Wang-Mascianica_et_al._2025}. 


\subsection{Coalgebraic point of view}\label{subs_coalg}
To illustrate the coalgebraic viewpoint on systems, let us use the example of Moore machines\footnote{a notion that generalizes that of deterministic finite automaton}. 

\begin{mdframed}{\textbf{Moore machines as coalgebras}}


	A Moore machine is defined by sets\footnote{More generally, objects.} of states, inputs and outputs, denoted respectively $S$, $I$, and $O$, and functions\footnote{More generally, morphisms.} output $S \rightarrow O$, and state update $S \times I \rightarrow S$.  
	
	
	The observation is then that, for $I$ and $O$ fixed, the data of a Moore machine is equivalent to the data of a set $S$, and a function $S \rightarrow O \times S^I$, that is, an object $S$ of the category of sets, equipped with a morphism $S \rightarrow F(S)$, where $F$ denotes the functor $X \mapsto O \times X^I$. In other words, a Moore machine is a coalgebra for the functor $F$. This observation is not restricted to the case of Moore machines, but extends to a large number of examples. See \cite[Section 3]{Rutten_2000}. 

\end{mdframed}



Starting from this principle, one can then define\footnote{It is more a matter of \emph{identifying a common structure} in systems theories, of \emph{organizing information}, than of conducting an in-depth axiomatic study from such definitions.} a system as being a coalgebra for a functor $F: \CC \rightarrow \CC$, it being understood that the notion strongly depends on $F$; in particular, the choice of $F$ determines a common "interface". A \emph{simulation} from one system to another is then simply a coalgebra morphism; in most examples, this notion translates the idea of "comparison morphism between state spaces, \emph{acting identically on interfaces}, compatible with the dynamics of the systems considered". More symmetric notions, of \emph{bisimulation} and \emph{bisimilarity}, can then be defined \cite{Staton_2009}.


%The coalgebraic viewpoint has given rise to a certain number of developments, notably at the logical level \cite{Kurz} \cite{Gallardo_Viglizzo_2024}; one of the medium-term objectives of my research program is to adapt, if this proves possible, these works to the doubly categorical framework.


\subsection{Synthetic Probability and Nondeterminism Theory, Markov Categories}\label{subs_th_synth_proba}

Before presenting doubly categorical systems theory, I wish to dwell on an important element at the "symmetric monoidal" level, namely \emph{synthetic probability theory}, based on the notion of Markov category \cite[Definition 2.1]{FRITZ-MarkovCats}. The idea is to axiomatize the behavior of \emph{Markov kernels} between measurable spaces, which associate to each element of the source a probability measure on the target space, which "varies in a measurable way" (see \cite[Section 1]{Perrone2022MarkovCA} and \cite{Giry_1982}). Since Markov kernels are equipped with parallel and sequential composition operations, \emph{the symmetric monoidal framework is relevant}. However, additional structure exists, specifically, for every measurable space $X$, the unique Markov kernel $del_X$, from $X$ to the singleton space, as well as the measurable diagonal function $copy_X: X \rightarrow X \times X$. Abstracting from this example:

\begin{mdframed}{\textbf{Definition:}}
A \emph{Markov category} is a symmetric monoidal category where every object $X$ is equipped with morphisms $del_X: X \rightarrow 1$ and $copy_X: X \rightarrow X \otimes X$, such that these morphisms satisfy a certain number of simple algebraic properties.
\end{mdframed}



Note that these axioms admit various models, some of which represent notions of \emph{possibilistic} nondeterminism, that is, where uncertainty corresponds to \emph{sets of possible outcomes}, without a probability measure to quantify it. The operation of associating to a (reasonable) probability measure its \emph{support} then induces a 
functor between the corresponding Markov categories, representing information loss. These considerations have importance for applications to the study of AI systems: a certain number of experimental results using less robust methodologies\footnote{Sometimes for lack of resources.}, or with sparse statistical analyses, can be considered as \emph{possibilistic}, and thus have conclusions in line with their methods.


An important point, as explained by Tobias Fritz in the introduction of \cite{FRITZ-MarkovCats}, is that this approach is \emph{axiomatic and synthetic}: it is about studying the \emph{behavior} of objects at a relatively abstract level, rather than relying on very precise (set-theoretic for example) descriptions. Among examples of successful synthetic approaches in mathematics, one can think of the theory of \emph{abelian categories}, which provides an efficient framework for treating questions of homological algebra, the theory of \emph{$\infty$-cosmoi} \cite{Riehl_Verity_2022}, which axiomatizes not $\infty$-categories themselves, but the 
universes in which they interact as objects, called $\infty$-cosmoi\footnote{This viewpoint allows treatment \emph{independent of the combinatorial model} chosen to define what an $\infty$-category is, which is important given the plurality of existing models.}. 


\subsection{Doubly Categorical Systems Theory}
As mentioned above, doubly categorical systems theory aims to combine a viewpoint on the \emph{composition} of systems and notions of \emph{comparisons}, or \emph{generalized simulations} between systems. One of the objectives is to provide principles and methods, supported by theory, for \emph{collaborative modeling}; the most salient projects in this area are ModelCollab \cite{UofS-CEPHIL/modelcollab_2025} and CatColab \cite{Carlson_2024}.

At the technical level, this relies on the use of \emph{double categories}\footnote{See \url{ncatlab.org/nlab/show/double+category} or \cite[Introduction]{Dawson_Pare_1993} for more details}. 

\begin{mdframed}{\textbf{Double categories and generalized simulations:}}
\begin{itemize}
	\item 	A (strict) double category is defined by two categories on the same class of objects, sometimes called the \emph{vertical} and \emph{horizontal} categories, as well as the data of $2$-cells, for any quadruple of morphisms forming the parallel sides of a square. The $2$-cells are sometimes called \emph{squares}, and are equipped with vertical and horizontal composition operations, associative and unital, satisfying the interchange law. 

	\item 	Given systems $S$ and $T$, viewed as $1$-dimensional cells in the double category at play, a \emph{behavior} of shape $S$ in $T$, or \emph{generalized simulation} from $S$ to $T$, or \emph{system morphism} from $S$ to $T$, is defined as being a $2$-cell. Thus, the principle is to construct double categories where one direction encodes system composition, and the other generalized simulations between interfaces, and between systems. \emph{The interchange law then represents a crucial compatibility condition between system composition and composition of generalized simulations}.

	
	\item An important idea is that each system \emph{represents} a type of behavior, and a question (cf \cite[End of Section 3.5]{DCST-book}) is to find (necessarily simple) systems representing the notions of \emph{trajectories}, nondeterministic in this case.
	\item 	Note that the notion of double category is more general than the more common notion of $2$-category\footnote{See \url{ncatlab.org/nlab/show/2-category}}; the difference is not anecdotal, because double categories allow \emph{consideration of generalized simulations between systems not having the same interface}.

\end{itemize}

	




	

\end{mdframed}








\subsection{Questions}
\begin{mdframed}{\textbf{Questions:}}
\begin{itemize}
	\item How to define a notion of "generalized simulation" between nondeterministic systems (stochastic, possibilistic, etc.) that can include nondeterminism, in a doubly categorical framework?
	\item How to answer the first question parametrically/functorially in the notion of nondeterminism?
	\item How to capture, in a uniform definition, discrete-time and continuous-time systems?
\end{itemize}

\end{mdframed}



\subsection{Existing Work}
\begin{itemize}
	\item In \cite{Baez_Courser_2018}, a (symmetric monoidal) double category is constructed, where one direction encodes open Markov processes, thus probabilistic systems, and the other coarse-graining functions. The main restriction is that these functions are deterministic. Another point is the focus on sequential and parallel compositions, at the expense of more general composition schemes.
	
    \item 

	\item In \cite{DCST-book} and \cite{Libkind_Myers_2025}, the constructions use the notion of \emph{commutative monad} for nondeterminism; they indeed give double categories, but the notion of generalized simulation is too restrictive, as explained in \cite[End of section 3.5]{DCST-book}.

\end{itemize}

\subsection{Contributions}

In \cite{Wang_2025}, I construct, given a Markov category with conditional laws\footnote{This condition being the synthetic version of the existence of conditional laws in the usual sense.}, denoted $\CC$, and a directed acyclic graph $\GG$, a dynamical systems theory, in a sense very close to what is called "systems module\footnote{Up to the absence of strict identities for one of the directions; I obtain what I call systems \emph{semi}modules.}" in \cite{Libkind_Myers_2025}, which answers the question posed in \cite[End of Section 3.5]{DCST-book}, i.e. which allows trajectories, and more generally behaviors, that are "truly nondeterministic". 


This construction allows capturing classes of examples previously not covered, or only restricted to deterministic behaviors, by doubly categorical systems theory:

\begin{itemize}
	\item (Open) systems governed by Stochastic Differential Equations.
	\item Nondeterministic automata.
	\item Partially observable Markov decision processes (i.e. Markov processes that can interact with an external environment).
\end{itemize}



At the technical level, my constructions rely on a certain number of ideas.

\begin{mdframed}{\textbf{Ideas:}}
	
\begin{enumerate}
	\item The Markov category $\CC$ with conditional laws represents the notion of nondeterminism of the systems theory to be constructed.
	\item The graph $\GG$ represents the notion of time considered; for reasons specific to nondeterminism\footnote{Precisely, the fact that a law on a product contains more information than the data of its marginals.}, and by absence of relevant closed monoidal or Cartesian closed structure\footnote{That is, where morphism spaces would be represented by objects of the category itself.}, I treat time externally to $\CC$. 
	\item The existence of conditional laws in $\CC$ allows defining the vertical composition of $2$-cells, which corresponds to \emph{constructing joint trajectories of composed systems}, by making \emph{an assumption of conditional independence of components relative to information at interfaces}. 
	\item From the idea of the point above, the essential technical work consists in verifying the required algebraic properties; associativity and interchange require the most effort.
	\item In order to simplify the writing, as well as future proofs of functoriality of the construction with respect to $\CC$ and $\GG$, I first construct \emph{triple categories}, where the additional dimension serves to manage time, before deducing the desired double categories as functor categories "with source $\GG$" valued in these triple categories.
\end{enumerate}
\end{mdframed}


An interesting observation is that the bulk of the calculations relies on using formal properties of conditional independence, and that said properties are very similar to the properties of independence notions used in my previous work, in model theory. This is not a coincidence: according to the article \cite{Yaacov_2013}, conditional independence of random variables (valued in a metric space) is an instance, in continuous model theory, of the general independence notions present in my work in model theory.


\printbibliography[
heading=bibintoc,
title={Bibliography}
]

    
\end{document}